% -*- TeX -*- -*- UK -*-
% ----------------------------------------------------------------
% arXiv Paper ************************************************
%
% Subhaneil Lahiri's template
%
% Before submitting:
%    Comment out hyperref
%    Comment out showkeys
%    Comment out natbib
%    Uncomment cite
%    Replace \newcommand{\mlim}[2]{{\stackrel{\scriptstyle #1}{#2}}}
\newcommand{\ra}{\rightarrow}
\newcommand{\lr}{\leftrightarrow}
\newcommand{\cdt}{\!\cdot\!}
\newcommand{\vp}{\vspace{0.5cm}}
\newcommand{\degs}{^\circ}
%
%e.g., i.e. with normal spaces
\newcommand{\eg}{e.g.\ }
\newcommand{\ie}{i.e.\ }
\newcommand{\cf}{cf.\ }
\newcommand{\etc}{etc.\ }
%
% indices
\newcommand{\up}[1]{\mbox{}^{#1}}
\newcommand{\dn}[1]{\mbox{}_{#1}}
\newcommand{\rp}[1]{^{(#1)}}
\newcommand{\lp}[1]{_{(#1)}}
%
% brackets etc.
\newcommand{\prn}[1]{\left ( #1 \right )}
\newcommand{\brc}[1]{\left\{ #1 \right\}}
\newcommand{\brk}[1]{\left [ #1 \right ]}
\newcommand{\abs}[1]{\left\lvert #1 \right\rvert}
\newcommand{\nrm}[1]{\left\lVert #1 \right\rVert}
\newcommand{\av}[1]{\left\langle #1 \right\rangle}
%
% QM Dirac notation
\newcommand{\bra}[1]{\left\langle #1 \right \rvert}
\newcommand{\ket}[1]{\left \lvert #1 \right\rangle}
\newcommand{\braket}[2]{\left\langle #1 \midddle | #2 \right\rangle}
\newcommand{\bracket}[3]{\left\langle #1 \middle | #2 \middle | #3 \right\rangle}
%
% Derivatives, etc. First argument is optional.
\newcommand{\diff}[3][\rule{0mm}{0mm}]{\frac{\mathrm{d}^{#1} #2}{\mathrm{d}{#3}^{#1}}}
\newcommand{\pdiff}[3][\rule{0mm}{0mm}]{\frac{\partial^{#1} #2}{\partial {#3}^{#1}}}
\newcommand{\pdiffc}[3][\rule{0mm}{0mm}]{\left (\frac{\partial #2}{\partial {#3}}\right )_{\!\!#1}}
\newcommand{\pdl}[1][\rule{0mm}{0mm}]{\overleftarrow{\partial}_{#1}}
\newcommand{\pdr}[1][\rule{0mm}{0mm}]{\overrightarrow{\partial}_{#1}}
\newcommand{\pdlr}[1][\rule{0mm}{0mm}]{\overleftrightarrow{\partial_{#1}}}
\newcommand{\fdf}[2]{\frac{\delta #1}{\delta #2}}
\newcommand{\intd}[1]{\int\!\dr #1\,}
%
% Un-italicised letters
\newcommand{\dr}{\mathrm{d}}
\newcommand{\e}{\mathrm{e}}
\newcommand{\ir}{\mathrm{i}}
\DeclareMathOperator{\tr}{tr}
\DeclareMathOperator{\Tr}{Tr}
\DeclareMathOperator{\Det}{Det}
%
% The default \Im and \Re look crap
\renewcommand{\Im}{\operatorname{\mathfrak{Im}}}
\renewcommand{\Re}{\operatorname{\mathfrak{Re}}}
%
% Referencing sections, figures, etc
\newcommand{\sref}[1]{\S\ref{#1}}
\newcommand{\cref}[1]{Ch.\ref{#1}}
\newcommand{\Cref}[1]{Ch.\ref{#1}}
\newcommand{\fref}[1]{fig.\ref{#1}}
\newcommand{\Fref}[1]{Fig.\ref{#1}}
\newcommand{\tref}[1]{tab.\ref{#1}}
\newcommand{\Tref}[1]{Tab.\ref{#1}}
%
\newcommand{\nn}{\nonumber}
%
% Put the preprint numbers in the top right corner of the page.
% Use after \maketitle.
% First argument: How high it needs to be raised,
% Second argument: Width of the box,
% Third argument: The preprint numbers.
\newcommand{\preprintno}[3]{\hfill\raisebox{#1}[0cm][0cm]{
\begin{minipage}[t]{#2}\begin{flushright} #3 \end{flushright}\end{minipage}}
\vspace*{-\baselinestretch\baselineskip}}
%
% If you have changed the line spacing, e.g. with \renewcommand{\baselinestretch}{1.5},
% the command \sgap produces a line break with the normal spacing.
\newlength{\lingap}
\setlength{\lingap}{\baselinestretch\baselineskip}
\addtolength{\lingap}{-\baselineskip}
\newcommand{\sgap}{\\[-\lingap]}
 with its contents
%    Replace %
\newcommand{\CD}{\mathcal{D}}
\newcommand{\CE}{\mathcal{E}}
\newcommand{\CG}{\mathcal{G}}
\newcommand{\CH}{\mathcal{H}}
\newcommand{\CK}{\mathcal{K}}
\newcommand{\CO}{\mathcal{O}}
\newcommand{\CL}{\mathcal{L}}
\newcommand{\CM}{\mathcal{M}}
\newcommand{\CN}{\mathcal{N}}
\newcommand{\CV}{\mathcal{V}}
\newcommand{\CZ}{\mathcal{Z}}
%
\newcommand{\dM}{\mathfrak{M}}
\newcommand{\dmd}{\mathfrak{d}}
\newcommand{\dmD}{\mathfrak{D}}
%
\newcommand{\R}{\mathbb{R}}
\newcommand{\C}{\mathbb{C}}
\newcommand{\CP}{\mathbb{CP}}
\newcommand{\Z}{\mathbb{Z}}
%
\newcommand{\ad}{{\dot{\alpha}}}
\newcommand{\bd}{{\dot{\beta}}}
\newcommand{\gd}{{\dot{\gamma}}}
\newcommand{\dd}{{\dot{\delta}}}
\newcommand{\ed}{{\dot{\epsilon}}}
%
\newcommand{\bs}{\overline{\sigma}}
\newcommand{\br}{\overline{\rho}}
\newcommand{\bpsi}{\overline{\psi}}
\newcommand{\bchi}{\overline{\chi}}
\newcommand{\bPsi}{\overline{\Psi}}
\newcommand{\bQ}{\overline{Q}}
\newcommand{\bS}{\overline{S}}
\newcommand{\bJ}{\overline{J}}
\newcommand{\zb}{{\bar z}}
\newcommand{\wb}{{\overline w}}
\newcommand{\cb}{{\bar c}}
\newcommand{\ab}{{\bar a}}
\newcommand{\bb}{{\bar b}}
\newcommand{\bp}{{\bar\partial}}
%
\newcommand{\p}{\partial}
\newcommand{\apm}{{\alpha^{\prime}}}
\newcommand{\adg}{a^\dagger}
\newcommand{\psq}{^{\prime\,2}}
\newcommand{\ppsq}{^{\prime\prime\,2}}
\newcommand{\half}{\frac{1}{2}}
%
 with its contents
%    Put this file, the .bbl file, any picture or
%       other additional files and Beisert's cite.sty
%       file in a zip/tar file
%
% **** -----------------------------------------------------------
\documentclass[12pt]{article}
% Preamble:
\usepackage{a4wide}
\usepackage[centertags]{amsmath}
\usepackage{amssymb}
%\usepackage{amsthm}
\usepackage[sort&compress,numbers]{natbib}
%\usepackage{citeB}
\usepackage{ifpdf}
%\usepackage{graphicx}
%\usepackage{graphics} for finding documentation only
%\usepackage{xcolor}
%\usepackage{pgf}
\ifpdf
\usepackage[pdftex,bookmarks]{hyperref}
\else
\usepackage[hypertex]{hyperref}
\fi
%
% >> Only for drafts! <<
\usepackage[notref,notcite]{showkeys}
% ----------------------------------------------------------------
\vfuzz2pt % Don't report over-full v-boxes if over-edge is small
\hfuzz2pt % Don't report over-full h-boxes if over-edge is small
%\numberwithin{equation}{section}
%\renewcommand{\baselinestretch}{1.5}
% ----------------------------------------------------------------
% New commands etc.
\newcommand{\mlim}[2]{{\stackrel{\scriptstyle #1}{#2}}}
\newcommand{\ra}{\rightarrow}
\newcommand{\lr}{\leftrightarrow}
\newcommand{\cdt}{\!\cdot\!}
\newcommand{\vp}{\vspace{0.5cm}}
\newcommand{\degs}{^\circ}
%
%e.g., i.e. with normal spaces
\newcommand{\eg}{e.g.\ }
\newcommand{\ie}{i.e.\ }
\newcommand{\cf}{cf.\ }
\newcommand{\etc}{etc.\ }
%
% indices
\newcommand{\up}[1]{\mbox{}^{#1}}
\newcommand{\dn}[1]{\mbox{}_{#1}}
\newcommand{\rp}[1]{^{(#1)}}
\newcommand{\lp}[1]{_{(#1)}}
%
% brackets etc.
\newcommand{\prn}[1]{\left ( #1 \right )}
\newcommand{\brc}[1]{\left\{ #1 \right\}}
\newcommand{\brk}[1]{\left [ #1 \right ]}
\newcommand{\abs}[1]{\left\lvert #1 \right\rvert}
\newcommand{\nrm}[1]{\left\lVert #1 \right\rVert}
\newcommand{\av}[1]{\left\langle #1 \right\rangle}
%
% QM Dirac notation
\newcommand{\bra}[1]{\left\langle #1 \right \rvert}
\newcommand{\ket}[1]{\left \lvert #1 \right\rangle}
\newcommand{\braket}[2]{\left\langle #1 \midddle | #2 \right\rangle}
\newcommand{\bracket}[3]{\left\langle #1 \middle | #2 \middle | #3 \right\rangle}
%
% Derivatives, etc. First argument is optional.
\newcommand{\diff}[3][\rule{0mm}{0mm}]{\frac{\mathrm{d}^{#1} #2}{\mathrm{d}{#3}^{#1}}}
\newcommand{\pdiff}[3][\rule{0mm}{0mm}]{\frac{\partial^{#1} #2}{\partial {#3}^{#1}}}
\newcommand{\pdiffc}[3][\rule{0mm}{0mm}]{\left (\frac{\partial #2}{\partial {#3}}\right )_{\!\!#1}}
\newcommand{\pdl}[1][\rule{0mm}{0mm}]{\overleftarrow{\partial}_{#1}}
\newcommand{\pdr}[1][\rule{0mm}{0mm}]{\overrightarrow{\partial}_{#1}}
\newcommand{\pdlr}[1][\rule{0mm}{0mm}]{\overleftrightarrow{\partial_{#1}}}
\newcommand{\fdf}[2]{\frac{\delta #1}{\delta #2}}
\newcommand{\intd}[1]{\int\!\dr #1\,}
%
% Un-italicised letters
\newcommand{\dr}{\mathrm{d}}
\newcommand{\e}{\mathrm{e}}
\newcommand{\ir}{\mathrm{i}}
\DeclareMathOperator{\tr}{tr}
\DeclareMathOperator{\Tr}{Tr}
\DeclareMathOperator{\Det}{Det}
%
% The default \Im and \Re look crap
\renewcommand{\Im}{\operatorname{\mathfrak{Im}}}
\renewcommand{\Re}{\operatorname{\mathfrak{Re}}}
%
% Referencing sections, figures, etc
\newcommand{\sref}[1]{\S\ref{#1}}
\newcommand{\cref}[1]{Ch.\ref{#1}}
\newcommand{\Cref}[1]{Ch.\ref{#1}}
\newcommand{\fref}[1]{fig.\ref{#1}}
\newcommand{\Fref}[1]{Fig.\ref{#1}}
\newcommand{\tref}[1]{tab.\ref{#1}}
\newcommand{\Tref}[1]{Tab.\ref{#1}}
%
\newcommand{\nn}{\nonumber}
%
% Put the preprint numbers in the top right corner of the page.
% Use after \maketitle.
% First argument: How high it needs to be raised,
% Second argument: Width of the box,
% Third argument: The preprint numbers.
\newcommand{\preprintno}[3]{\hfill\raisebox{#1}[0cm][0cm]{
\begin{minipage}[t]{#2}\begin{flushright} #3 \end{flushright}\end{minipage}}
\vspace*{-\baselinestretch\baselineskip}}
%
% If you have changed the line spacing, e.g. with \renewcommand{\baselinestretch}{1.5},
% the command \sgap produces a line break with the normal spacing.
\newlength{\lingap}
\setlength{\lingap}{\baselinestretch\baselineskip}
\addtolength{\lingap}{-\baselineskip}
\newcommand{\sgap}{\\[-\lingap]}

%
\newcommand{\CD}{\mathcal{D}}
\newcommand{\CE}{\mathcal{E}}
\newcommand{\CG}{\mathcal{G}}
\newcommand{\CH}{\mathcal{H}}
\newcommand{\CK}{\mathcal{K}}
\newcommand{\CO}{\mathcal{O}}
\newcommand{\CL}{\mathcal{L}}
\newcommand{\CM}{\mathcal{M}}
\newcommand{\CN}{\mathcal{N}}
\newcommand{\CV}{\mathcal{V}}
\newcommand{\CZ}{\mathcal{Z}}
%
\newcommand{\dM}{\mathfrak{M}}
\newcommand{\dmd}{\mathfrak{d}}
\newcommand{\dmD}{\mathfrak{D}}
%
\newcommand{\R}{\mathbb{R}}
\newcommand{\C}{\mathbb{C}}
\newcommand{\CP}{\mathbb{CP}}
\newcommand{\Z}{\mathbb{Z}}
%
\newcommand{\ad}{{\dot{\alpha}}}
\newcommand{\bd}{{\dot{\beta}}}
\newcommand{\gd}{{\dot{\gamma}}}
\newcommand{\dd}{{\dot{\delta}}}
\newcommand{\ed}{{\dot{\epsilon}}}
%
\newcommand{\bs}{\overline{\sigma}}
\newcommand{\br}{\overline{\rho}}
\newcommand{\bpsi}{\overline{\psi}}
\newcommand{\bchi}{\overline{\chi}}
\newcommand{\bPsi}{\overline{\Psi}}
\newcommand{\bQ}{\overline{Q}}
\newcommand{\bS}{\overline{S}}
\newcommand{\bJ}{\overline{J}}
\newcommand{\zb}{{\bar z}}
\newcommand{\wb}{{\overline w}}
\newcommand{\cb}{{\bar c}}
\newcommand{\ab}{{\bar a}}
\newcommand{\bb}{{\bar b}}
\newcommand{\bp}{{\bar\partial}}
%
\newcommand{\p}{\partial}
\newcommand{\apm}{{\alpha^{\prime}}}
\newcommand{\adg}{a^\dagger}
\newcommand{\psq}{^{\prime\,2}}
\newcommand{\ppsq}{^{\prime\prime\,2}}
\newcommand{\half}{\frac{1}{2}}
%

\newcommand{\tloc}{\mathcal{T}}
\newcommand{\floc}{\mathcal{F}}
\newcommand{\eloc}{\mathcal{E}}
\newcommand{\sloc}{\mathcal{S}}
\newcommand{\ploc}{\mathcal{P}}
\newcommand{\rloc}{\mathcal{R}}
\newcommand{\rl}{\mathfrak{r}}
\newcommand{\ml}{\mathfrak{m}}
\newcommand{\mg}{\mu}
\newcommand{\gpf}{\mathcal{Z}_\mathrm{gc}}
\newcommand{\mfp}{l_{\mathrm{mfp}}}
\newcommand{\eos}{\alpha}
\newcommand{\tc}{\mathcal{T_\mathrm{c}}}
%\newcommand{\rz}{\rho_0}
\newcommand{\rc}{\rho_\mathrm{c}}
\newcommand{\cs}{c_\mathrm{s}}
\newcommand{\vint}{\int_\Gamma\!\dr^{d-1}x}
\newcommand{\aint}{\int_{\p\Gamma}\!\!\dr^{d-2}x}
\newcommand{\dvint}{\int_{\Delta\Gamma}\!\dr^{d-1}x}
\newcommand{\daint}{\int_{\p\Delta\Gamma}\!\!\dr^{d-2}x}
%
%%%%%%%%%%%%%%%%%%%%%%%%%%%%%%%%%%%%%%%%%%%%%%%%%%%%%%%%%%%%%%%%%%%%%%%%%%
% Title info:
\title{Equilibrium configurations}
%
% Author List:
%
\author{Subhaneil Lahiri
\\
%
% Addresses:
%
\small{\emph{Jefferson Physical Laboratory,
                   Harvard University, Cambridge MA 02138, USA}}
%
}

\begin{document}

\maketitle

% Preprint numbers, etc.
\preprintno{8cm}{6cm}{
    \texttt{arXiv:yymm.nnnn [hep-th]}
}

%%%%%%%%%%%%%%%%%%%%%%%%%%%%%%%%%%%%%%%%%%%%%%%%%%%%%%%%%%%%%%%%%%%%%%%%%%


\begin{abstract}
  We look at equilibrium configurations of relativistic fluids.
\end{abstract}


%%%%%%%%%%%%%%%%%%%%%%%%%%%%%%%%%%%%%%%%%%%%%%%%%%%%%%%%%%%%%%%%%%%%%%%%%%
% Beginning of Article:
%%%%%%%%%%%%%%%%%%%%%%%%%%%%%%%%%%%%%%%%%%%%%%%%%%%%%%%%%%%%%%%%%%%%%%%%%%

\section{Thermodynamics}\label{sec:fltherm}

A fluid satisfies
%
\begin{equation}\label{firstlaw:eq}
  \dr\eloc = \tloc\dr\sloc - \ploc\dr \mathcal{V} + \ml_i \,\dr\rloc_i.
\end{equation}
%
Suppose we rescale the system by a factor $(1+\epsilon)$.
Extensivity tells us that $\dr\eloc=\epsilon\eloc$, $\dr\sloc=\epsilon\sloc$, $\dr\mathcal{V}=\epsilon\mathcal{V}$ and $\dr\rloc_i=\epsilon\rloc_i$.
Then \eqref{firstlaw:eq} tells us that
%
\begin{equation*}
  \eloc = \tloc\sloc - \ploc \mathcal{V} + \ml_i \rloc_i.
\end{equation*}
%
Defining intensive quantities: density $\rho=\eloc/\mathcal{V}$, entropy density $s=\sloc/\mathcal{V}$ and charge density $\rl_i=\rloc/\mathcal{V}$, we have
%
\begin{equation}\label{inttherm:eq}
  \begin{split}
    \rho+\ploc &= s\tloc + \ml_i \rl_i, \\
    \dr\rho &= \tloc \dr s + \ml_i \,\dr \rl_i, \\
    \dr\ploc &= s\, \dr\tloc + \rl_i \,\dr\ml_i.
  \end{split}
\end{equation}
%
Note that all intensive thermodynamic quantities can be written as functions of $(1+c)$ variables, which we will usually choose to be the temperature, $\tloc$ and chemical potentials $\ml_i$. Once we are given the pressure as a function of temperature and chemical potential, we can use \eqref{inttherm:eq} to determine the others.


\section{Fluid mechanics}\label{sec:stress}

\subsection{The equations}\label{sec:basiceq}

Provided all length scales are large compared to the thermalisation scale of the fluid (which we call $\mfp$), each patch of the fluid is well described by equilibrium thermodynamics in its rest frame. The fluid is characterised by the velocity of these patches --- described by a vector $u^\mu=\gamma(1,\vec{v})$ --- and the intensive thermodynamic quantities in their rest frames --- which can all be computed from the proper temperature $\tloc$ and $\ml_i$ using the equation of state and the first law of thermodynamics, as in \S\ref{sec:fltherm}.

The equations of fluid dynamics are simply a statement of the conservation of the stress tensor $T^{\mu \nu}$ and the charge currents $J^\mu_i$.
%
\begin{equation}\label{Epconsv:eq} \begin{split}
  \nabla_\mu T^{\mu\nu} &%= \p_\mu T^{\mu\nu}
%                        + \Gamma^\mu_{\mu\lambda} T^{\lambda\nu}
%                        + \Gamma^\nu_{\mu\lambda} T^{\mu\lambda}
                        = 0\,, \\
\nabla_\mu J^\mu_i &%= \p_\mu J_i^\mu + \Gamma^\mu_{\mu \alpha} J^\alpha_i
                        =0\,.
\end{split}
\end{equation}
%
These provide $(d+c)$ equations for the evolution of for the $(d+c)$ quantities $\vec{v}$, $\tloc$ and $\mu_i$.

\subsection{Perfect fluid stress tensor}\label{sec:perfstr}

The dynamics of a fluid is completely specified once the stress tensor and charge currents are given as functions of $\tloc$, $\mu_i$ and $u^\mu$. As we have explained in the introduction, fluid mechanics is an effective description at long distances (i.e, it is valid only when the fluid variables vary on distance scales that are large compared to the mean free path $l_\mathrm{mfp}$). As a consequence it is natural to expand the stress tensor and charge current in powers of derivatives. In this subsection we briefly review the leading (i.e.\ zeroth) order terms in this expansion.

It is convenient to define a projection tensor
%
\begin{equation}\label{proj:eq}
  P^{\mu\nu} = g^{\mu\nu} + u^\mu u^\nu.
\end{equation}
%
$P^{\mu\nu}$ projects vectors onto the $(d-1)$ dimensional submanifold orthogonal to $u^\mu$. In other words, $P^{\mu\nu}$ may be thought of as a projector onto spatial coordinates in the rest frame of the fluid. In a similar fashion, $- u^\mu u^\nu$ projects vectors onto the time direction in the fluid rest frame.

To zeroth order in the derivative expansion, Lorentz invariance and the correct static limit uniquely determine the stress tensor, charge and the entropy currents in terms of the thermodynamic variables. We have
%
\begin{equation}\label{currents:eq}
\begin{split}
  T^{\mu\nu}_\mathrm{perfect}& = \rho u^\mu u^\nu + \ploc P^{\mu\nu}, \\
  (J^\mu_i)_\mathrm{perfect}&=\rl_iu^\mu, \\
  (J^\mu_S)_\mathrm{perfect}&=s u^\mu,
\end{split}
\end{equation}
%
where all thermodynamic quantities are measured in the local rest frame of the fluid, so that they are Lorentz scalars. It is not difficult to verify that in this zero-derivative (or perfect fluid) approximation, the entropy current is conserved. Entropy production (associated with dissipation) occurs only at the first subleading order in the derivative expansion, as we will see in the next subsection.

\subsection{Dissipation and diffusion}\label{sec:visc}

Now, we proceed to examine the first subleading order in the derivative expansion. In the first subleading order, Lorentz invariance and the physical requirement that entropy be non-decreasing determine the form of the stress tensor and the current to be (see, for example, \S\S14.1 of \cite{Andersson:2006nr})
%
\begin{equation}\label{extraTvisc:eq}
\begin{split}
  T^{\mu\nu}_\mathrm{dissipative} &= -\zeta \vartheta P^{\mu\nu} -
  2\eta\sigma^{\mu\nu} + q^\mu u^\nu + u^\mu q^\nu,\\
 (J^\mu_i)_\mathrm{dissipative} &= j^\mu_{i},\\
(J^\mu_S)_\mathrm{dissipative} &= \frac{q^\mu-\ml_i j^\mu_{i}}{\tloc}\,.
\end{split}
\end{equation}
%
where
%
\begin{equation}\label{fluidtensors:eq}
\begin{split}
  a^\mu &= u^\nu \nabla_\nu u^\mu, \\
  \vartheta &= \nabla_\mu u^\mu, \\
%  P^{\mu\nu} &= g^{\mu\nu} + u^\nu u^\mu, \\
  \sigma^{\mu\nu} &= \half \prn{P^{\mu\lambda} \nabla_\lambda u^\nu
                   + P^{\nu\lambda} \nabla_\lambda u^\mu}
                   - \frac{1}{d-1} \vartheta P^{\mu\nu}, \\
  \omega^{\mu\nu} &= \half \prn{P^{\mu\lambda} \nabla_\lambda u^\nu
                   - P^{\nu\lambda} \nabla_\lambda u^\mu} \\
 q^\mu &= -\kappa P^{\mu\nu} (\p_\nu\tloc + a_\nu\tloc)\,, \\
 j^\mu_{i}
   & = - D_{ij} P^{\mu\nu}\p_\nu\! \prn{\frac{\ml_j}{\tloc}},
\end{split}
\end{equation}
%
are the acceleration, expansion, shear tensor, vorticity
heat flux and diffusion current respectively.
We can write
%
\begin{equation}\label{velder:eq}
  \nabla_\mu u_\nu = \sigma_{\mu\nu}+\omega_{\mu\nu}+\frac{\vartheta}{d-1}P_{\mu\nu}-u_\mu a_\nu.
\end{equation}
%


These equations define a set of new fluid dynamical parameters in addition to those of the previous subsection: $\zeta$ is the bulk viscosity, $\eta$ is the shear viscosity, $\kappa$ is the thermal conductivity and $D_{ij}$ are the diffusion coefficients. Fourier's law of heat conduction $\vec{q} = -\kappa \vec{\nabla} \tloc$ has been relativistically modified to
%
\begin{equation}\label{heatcond:eq}
  q^\mu = -\kappa P^{\mu\nu} (\p_\nu\tloc + a_\nu\tloc)\,,
\end{equation}
%
with an extra term that is related to the redshifting of the temperature. The diffusive contribution to the charge current is the relativistic generalisation of Fick's law.

At this order in the derivative expansion, the entropy current is no longer conserved; however, it may be checked \cite{Andersson:2006nr} that
%
\begin{equation}\label{increase:eq}
\tloc\nabla_\mu J^\mu_S = \frac{q^\mu q_\mu}{\kappa \tloc}
+ \tloc (D^{-1})^{ij} j_i^\mu j_{j\mu}  + \zeta \theta ^2
+ 2 \eta \sigma_{\mu \nu} \sigma^{\mu \nu}.
\end{equation}
%
As $q^\mu$, $j^\mu_i$ and $\sigma^{\mu \nu}$ are all spacelike
vectors and tensors, the RHS of \eqref{increase:eq} is positive provided
$\eta, \zeta, \kappa$ and $D$ are positive parameters, a condition we further assume. This establishes that (even locally) entropy can only be produced but never destroyed. In equilibrium, $\nabla_\mu J^\mu_S$ must vanish. It follows that, $q^\mu$, $j^\mu_i$, $\theta$ and $\sigma^{\mu \nu}$
each individually vanish in equilibrium.

For fluids with gravity duals, the shear viscosity is given by $\eta=\frac{s}{4\pi}$ \cite{Son:2007vk}. We can estimate the thermalisation length of the fluid by comparing coefficients at different orders in the derivative expansion
%
\begin{equation}\label{mfp:eq}
  \mfp \sim \frac{\eta}{\rho} = \frac{s}{4\pi\rho}.
\end{equation}
%
This length scale may plausibly be identified with the thermalisation length scale of the fluid. This may be demonstrated within the kinetic theory, where $\mfp$ is simply the mean free path of colliding molecules, but is expected to apply to more generally to any fluid with short range interactions.

%When studying fluids on curved manifolds (as we will proceed to do in this paper), one could add generally covariant terms, built out of curvatures, to the stress tensor. For instance, we could add a term proportional to $R^{\mu \nu}$ to the expression for $T^{\mu\nu}$. We will ignore all such terms in this paper for a reason we now explain. In all the solutions of fluid mechanics that we will study, the length scale over which fluid quantities vary is the same as the length scale of curvatures of the manifold. Any expression built out of a curvature contains at least two spacetime derivatives of the metric; it follows that any contribution to the stress tensor proportional to a curvature is effectively at least two orders subleading in the derivative expansion, and so is negligible compared to all the other terms we have retained in this paper.

\section{Surfaces}\label{sec:surface}

The plasma ball configurations we consider have a domain wall separating a bubble of the deconfined phase from the confined phase. As the density, pressure, etc.\ of the deconfined phase are a factor of $N^2$ larger than the confined phase, we can treat the confined phase as the vacuum and the domain wall as a surface bounding the deconfined fluid.

At surfaces, the density of the fluid changes too rapidly to be described by fluid mechanics. However, provided that we look at length scales much larger than the thickness of the surface, we can replace this region by a delta function localised piece of the stress tensor.

At these length scales, this stress tensor will depend on the direction of the surface, with dependance on its curvature being suppressed.

\subsection{Thermodynamics}\label{sec:surftherm}

The surface energy, area, entropy and charge will satisfy
%
\begin{equation}\label{firstlawsurf:eq}
  \dr\eloc_\mathcal{A} = \tloc\,\dr\sloc_\mathcal{A} + \sigma\,\dr\mathcal{A} + \ml_i\,\dr\rloc_{i,\mathcal{A}}\,.
\end{equation}
%
where $\sigma$ is the surface tension and $\mathcal{A}$ is the surface area.
Suppose we rescale the area by a factor $(1+\epsilon)$.
Extensivity tells us that $\dr\eloc_\mathcal{A}=\epsilon\eloc_\mathcal{A}$, $\dr\sloc_\mathcal{A}=\epsilon\sloc_\mathcal{A}$, $\dr\mathcal{A}=\epsilon\mathcal{A}$ and $\dr\rloc_{i,\mathcal{A}}\epsilon\rloc_{i,\mathcal{A}}$.
Then \eqref{firstlawsurf:eq} tells us that
%
\begin{equation*}
  \eloc_\mathcal{A} = \tloc\sloc_\mathcal{A} + \sigma\mathcal{A} + \ml_i\rloc_{i,\mathcal{A}}\,.
\end{equation*}
%
Defining the intensive quantities: surface energy density $\sigma_E=\eloc_\mathcal{A}/\mathcal{A}$, surface entropy density $\sigma_S=\sloc_\mathcal{A}/\mathcal{A}$ and  surface R-charge densities $\sigma_{R_i}=\rloc_{i,\mathcal{A}}/\mathcal{A}$, considerations similar to those leading to \eqref{inttherm:eq} lead to
%
\begin{equation}\label{surftherm:eq}
  \begin{split}
    \sigma_E &= \sigma + \tloc \sigma_S + \ml_i \,\sigma_{R_i},
    \\
    \dr\sigma &= -\sigma_S\, \dr\tloc -\sigma_{R_i} \dr\ml_i.
  \end{split}
\end{equation}
%
%However, the surface tension was only computed at $\tloc=\tc$, $\ml_i=0$ in \cite{Aharony:2005bm}, so we will ignore its temperature and chemical potential dependence. As we can see above, this is equivalent to setting $\sigma_S=\sigma_{R_i}=0$ and $\sigma_E=\sigma$.
Note that using a temperature and chemical potential independent surface tension is equivalent to setting $\sigma_S=\sigma_{R_i}=0$ and $\sigma_E=\sigma$.

From the form of the gravity solution, we would expect the thickness of the surface to be approximately
%
\begin{equation}\label{thick:eq}
  \xi \sim \frac{\sigma_E}{\rho} \sim \frac{\sigma_S}{s} \sim \frac{\sigma}{\ploc} \sim \frac{\sigma_{R_i}}{\rl_i}\,.
\end{equation}
%
In general, it will be of order $N^0$ and is similar to  $\mfp$ (if $8\pi$ can be considered similar to 1).

For the domain wall of \cite{Aharony:2005bm}, the thickness and surface tension are $6\times \frac{1}{2\pi \tc}$ and $\sigma=2.0 \times \frac{\pi^2 N^2 \tc^2}{2}$ respectively. If we use the estimate $\xi = \frac{\sigma}{\rc} = \frac{2.0}{\tc}$, which is pretty close to the thickness.

\subsection{Stress tensor and currents}\label{sec:surfstr}

Let's describe the location of the surface by a function $f(x)$ that is positive inside the fluid and has a first order zero on the surface.
%
\begin{equation}\label{fluidsurf:eq}
  T^{\mu\nu} = \theta(f)T^{\mu\nu}_\text{fluid} + \delta(f)T^{\mu\nu}_\text{surface}.
\end{equation}
%
At large length scales, as mentioned above, $T^{\mu\nu}_\text{surface}$ will only depend on the first derivative of $f$ and no higher derivatives.

If we demand invariance under reparameterisations of the function $f(x) \to g(x)f(x)$, where $g(x) > 0$, and that the surface moves at the velocity of the fluid
%
\begin{equation}\label{surfvel:eq}
  u^\mu \p_\mu f \delta(f) =0,
\end{equation}
%
the most general surface stress tensor we can have is (see \S2.3 of \cite{Lahiri:2007ae})
%
\begin{equation*}
    T^{\mu\nu}_\mathrm{surface}\delta(f) = \brk{A n^\mu n^\nu
                   +B u^\mu u^\nu
                   +C \prn{u^\mu n^\nu + n^\mu u^\nu}
                   +D g^{\mu\nu}}\sqrt{\p f\cdt\p f}\delta(f)\,.
\end{equation*}
%
%
where $n_\mu = -\p_\mu f / \sqrt{\p f \cdt \p f}$ is the outward pointing unit normal to the surface.

We can fix $A,B,C,D$ by looking at a fluid at rest, $u^\mu = (1,0,0,\ldots)$, with a surface $f(x)=x$
%
\begin{equation*}
  T^{\mu\nu}_\mathrm{surface} =
  \begin{pmatrix}
    B-D & -C   & 0  \\
    -C   & A+D & 0  \\
    0   & 0   & D  \\
  \end{pmatrix}
     \delta(x)=
  \begin{pmatrix}
    \sigma_E & 0 &  0   \\
    0 & 0 &  0   \\
    0 & 0 & -\sigma   \\
  \end{pmatrix}
     \delta(x).
\end{equation*}
%
This gives
%
\begin{equation}\label{surfstressgen:eq}
  T^{\mu\nu}_\text{surface} = \sqrt{\p f \cdt \p f} \brk{\sigma_E u^\mu u^\nu -\sigma(h^{\mu\nu} + u^\mu u^\nu)},
\end{equation}
%
where $h_{\mu\nu} = g_{\mu\nu} - n_\mu n_\nu$ is the induced metric of the surface.

Similar reasoning leads to
%
\begin{equation}\label{surfcurr:eq}
  \begin{split}
    J^\mu_i &= \theta(f)J^\mu_{i,\text{fluid}} + \delta(f)J^\mu_{i,\text{ surface}}\,,\\
    J^\mu_{i,\text{ surface}} &= \sqrt{\p f \cdt \p f} \sigma_{R_i} u^\mu.\\
    J^\mu_S &= \theta(f)J^\mu_{S,\text{fluid}} + \delta(f)J^\mu_{S,\text{ surface}}\,,\\
    J^\mu_{S,\text{ surface}} &= \sqrt{\p f \cdt \p f} \sigma_{S} u^\mu.
  \end{split}
\end{equation}
%


The factor of $\sqrt{\p f \cdt \p f}$ also has a simple interpretation: suppose we use a coordinate system where $f$ is one of the coordinates. Then
%
\begin{equation}\label{surfmeasure:eq}
  \sqrt{\p f \cdt \p f} = \sqrt{g^{ff}} = \sqrt{\frac{\det h}{\det g}}\,,
\end{equation}
%
which provides the correct change of integration measure for localisation to the surface. If we used some other coordinates, there'd be an extra Jacobian factor.

\subsection{Equations of motion}\label{sec:surfeom}

We have
%
\begin{equation}\label{surfeom:eq}
  \nabla_\mu T^{\mu\nu} = \theta(f)\nabla_\mu T^{\mu\nu}_\text{fluid} + \delta(f)(\p_\mu f) T^{\mu\nu}_\text{fluid} + \delta(f)\nabla_\mu T^{\mu\nu}_\text{surface} + \delta'(f)(\p_\mu f) T^{\mu\nu}_\text{surface}.
\end{equation}
%
Taking the derivative of \eqref{surfvel:eq} and contracting with $\nabla_\nu f$ gives
%
\begin{equation*}
  (\nabla^\nu u^\mu) (\nabla_\mu f) (\nabla_\nu f) \delta(f)
  + u^\mu (\nabla^\nu f) (\nabla_\nu \nabla_\mu f) \delta(f)
  + u^\mu (\nabla_\mu f) (\p f \cdt \p f) \delta'(f)
  = 0.
\end{equation*}
%
This can be used to eliminate the last term of \eqref{surfeom:eq}
%
\begin{multline}\label{surfeom2:eq}
  \nabla_\mu T^{\mu\nu} = \theta(f)\nabla_\mu T^{\mu\nu}_\text{fluid} +\\ \delta(f)\sqrt{\p f \cdt \p f}\brk{
  -n_\mu T^{\mu\nu}_\text{fluid} + \frac{\nabla_\mu T^{\mu\nu}_\text{surface}}{\sqrt{\p f \cdt \p f}}
  +(\sigma_E-\sigma)u^\nu\prn{\frac{u^\mu n^\lambda \nabla_\mu\nabla_\lambda f}{\sqrt{\p f \cdt \p f}} - n_\mu n_\lambda \nabla^\mu u^\lambda}
  }.
\end{multline}
%
So, in addition to the equation of motion \eqref{Epconsv:eq}, we also have the boundary conditions coming from the term in square brackets
\footnote{As we are only keeping terms to leading order in the derivative expansion for the surface, we will do the same for the fluid here.}
%
\begin{equation}\label{surfbc:eq}
  \left.
  -\ploc n^\mu + u^\mu u^\nu \p_\nu (\sigma_E-\sigma)
  +(\sigma_E-\sigma)\prn{ \frac{d-2}{d-1}\vartheta u^\mu  + a^\mu - u^\mu n_\lambda n_\nu \sigma^{\lambda\nu}}
  -h^{\mu\nu}\p_\nu\sigma + \sigma\Theta n^\mu
  \right\vert_{f=0} = 0
\end{equation}
%
where $\Theta$ is the trace of the extrinsic curvature of the surface, as seen from outside the fluid (see \S\ref{sec:extrinsic}).

Similar reasoning leads to
%
\begin{equation}\label{surfchbc:eq}
  \nabla_\mu J^\mu_i= \theta(f)\nabla_\mu J^\mu_{i,\text{fluid}}
   + \delta(f)\sqrt{\p f \cdt \p f} \prn{\frac{d-2}{d-1}\sigma_{R_i}\vartheta + u^\mu\p_\mu\sigma_{R_i} - \sigma_{R_i}n_\mu n_\lambda \sigma^{\mu\nu}},
\end{equation}
%
which leads to an additional boundary condition to go with \eqref{Epconsv:eq}.

Also
%
\begin{equation*}
  \nabla_\mu J^\mu_S= \theta(f)\nabla_\mu J^\mu_{S,\text{fluid}},
\end{equation*}
%
so there is no entropy production at the surface to leading order in the derivative expansion.

%If we have several disconnected surfaces, it is convenient to make the separation $f=\prod_i f_i$. As the surfaces are disconnected, the zero sets of the $f_i$ do not intersect. Also, the $f_i$ are all positive inside the fluid. Therefore, whenever one of the $f_i$ is negative or zero, all the others are positive. Luckily, \eqref{surfeom:eq} splits nicely
%%
%\begin{equation*}
%  \nabla_\mu T^{\mu\nu} = \prod_i \theta\prn{f_i}\nabla_\mu T^{\mu\nu}_\text{fluid} + \sum_i \delta(f_i)\brk{(\p_\mu f_i) T^{\mu\nu}_\text{fluid} + \nabla_\mu T^{\mu\nu}_\text{surface}(f_i)}.
%\end{equation*}
%%





\section{Rigid rotation}\label{sec:rigidrot}

\subsection{Solutions for the interior}\label{sec:rotint}

We want to find solutions of \eqref{Epconsv:eq} that are independent of time, which means we need to set \eqref{increase:eq} to zero.
This means we need velocity configurations that have zero expansion and shear.
As argued in \cite{Caldarelli:2008mv}, combined with the equations of fluid mechanics, this implies that the velocity must be proportional to a Killing vector.
In general, this would be a combination of a uniform boost and rigid rotation.
We can always boost to a frame where the centre of rotation is static and the rotation lies in the Cartan directions of the rotation group. This gives
%
\begin{equation}\label{rigidrot:eq}
  u = \gamma(\p_t + \Omega_a l_a),
\end{equation}
%
where $\Omega_a$ are the angular velocities and $l_a$ are a set of commuting rotational Killing vectors. The important feature is that the velocity is a normalisation factor times a Killing vector:
%
\begin{equation}\label{eqvel:eq}
  u^\mu = \gamma K^\mu, \qquad
  \gamma^2 K^\mu K_\mu = -1, \qquad
  \nabla_{(\mu} K_{\nu)} = 0.
\end{equation}
%
One can deduce that
%
\begin{equation}\label{rotvelder:eq}
  \theta = \sigma^{\mu\nu}=0, \qquad
  u^\mu \p_\mu \gamma = 0, \qquad
  a_\mu = -\frac{\p_\mu \gamma}{\gamma}\,.
\end{equation}
%
Which leads to
%
\begin{equation*}
  q^\mu = -\kappa \gamma P^{\mu\nu} \p_\nu \brk{\frac{\tloc}{\gamma}}, \qquad
  j^\mu_i = -D_{ij} P^{\mu\nu} \p_\nu \brk{\frac{\ml_j}{\tloc}}.
\end{equation*}
%
One can also show that
%
\begin{equation*}
\begin{split}
  \nabla_\mu T^{\mu\nu}_\mathrm{perfect} =&
    \gamma \prn{s P^{\nu\mu}
        + \brc{\tloc \pdiffc{s}{\tloc} + \ml_i \pdiffc{\rl_i}{\tloc}} u^\nu u^\mu}
       \p_\mu \brk{\frac{\tloc}{\gamma}}\\
   &+\gamma \prn{\rl_i P^{\nu\mu}
        + \brc{\tloc \pdiffc{s}{\ml_i} + \ml_j \pdiffc{\rl_j}{\ml_i}} u^\nu u^\mu}
       \p_\mu \brk{\frac{\ml_i}{\gamma}},\\
  \nabla_\mu J^\mu_{i,\text{perfect}} =&
   \gamma \pdiffc{\rl_i}{\tloc} u^\mu\p_\mu \brk{\frac{\tloc}{\gamma}}
   + \gamma \pdiffc{\rl_i}{\ml_j} u^\mu\p_\mu \brk{\frac{\ml_j}{\gamma}}
\end{split}
\end{equation*}
%
So the velocity configuration \eqref{rigidrot:eq} will be an equilibrium solution to the equations of motion provided that
%
\begin{equation}\label{rotsol:eq}
  \frac{\tloc}{\gamma} = T = \text{constant}, \qquad
  \frac{\ml_i}{\gamma} = \mg_i = \text{constant}, \qquad
  \frac{\ml_i}{\tloc} = \nu_i = \frac{\mg_i}{T} = \text{constant}.
\end{equation}
%
Using the equation of state and \eqref{inttherm:eq}, this determines all of the intensive thermodynamic quantities in the fluid.

\subsection{Solutions for surfaces}\label{sec:rotsurf}

The fluid configurations described in the previous subsection have $\sigma^{\mu\nu}=\vartheta=u^\mu\p_\mu\gamma=0$.
In addition, the quantities $\sigma$, $\sigma_E$ and $\sigma_{R_i}$ are functions of $\tloc$ and $\ml_i$, which in turn are proportional to $\gamma$.
Therefore, using \eqref{surftherm:eq} and \eqref{rotvelder:eq},
%
\begin{equation}\label{rotsigmagrad:eq}
  u^\mu\p_\mu\sigma=u^\mu\p_\mu\sigma_E=u^\mu\p_\mu\sigma_{R_i}=0\,,
  \qquad
  \p_\mu \sigma = (\sigma_E - \sigma)a_\mu\,.
\end{equation}
%
This means that \eqref{surfbc:eq} and \eqref{surfchbc:eq} reduce to
%
\begin{equation}\label{rotsurfbc:eq}
  \ploc|_{f=0} = \sigma \Theta + (\sigma_E-\sigma)n\cdt a.
\end{equation}
%
As the pressure is determined by \eqref{rotsol:eq}, this provides a differential equation that determines allowed positions of surfaces. Demanding that the surface has no conical singularities turns out to provide enough boundary conditions to determine the position of the surface completely (up to discrete choices) in terms of the parameters $\Omega_a$, $T$ and $\mg_i$.

\subsection{Thermodynamics of solutions}\label{sec:rottherm}

In this section. we will use the symbol $\dr$ to indicate the change in a quantity due to a small change in the thermodynamic state of the system, \ie it is the exterior derivative on the space of thermodynamic states and not the space-time exterior derivative. The only exception will be the integration measures over the region occupied by the fluid, $\Gamma$, and the surface, $\p\Gamma$:
%
\begin{equation*}
  \vint\brk{\ldots}, \qquad \aint\brk{\ldots}.
\end{equation*}
%


We compute the extensive thermodynamic properties of these solutions by integrating the time components of the corresponding currents (noting that the current associated with a Killing vector $\zeta^\mu$ is $J^\mu_\zeta = T^{\mu\nu}\zeta_\nu$):
%
\begin{equation}\label{noetherch:eq}
 \begin{split}
  Q_X &= \vint J^0_X,
 \end{split}
\end{equation}
%
In particular, also noting that for equilibrium configurations $\p^0\!f=0$,
%
\begin{equation}\label{killingcharge:eq}
  Q_\zeta = \vint\brk{(\rho+\ploc)\gamma^2 K^0 K\cdt\zeta
   + \ploc \zeta^0 } + \aint \brk{(\sigma_E-\sigma)\gamma^2 K^0 K\cdt\zeta - \sigma \zeta^0}.
\end{equation}
%
Noting that $K^0=(\p_t)^0=1$ and $l_a^0=0$, this gives
%
\begin{equation}\label{thermcharge:eq}
  \begin{aligned}
    E &= -Q_{\p_t}& &= -\vint\brk{(\rho+\ploc)\gamma^2 K\cdt\p_t +\ploc}
       -\aint \brk{(\sigma_E-\sigma)\gamma^2 K\cdt\p_t -\sigma}, \\
    L_a &= Q_{l_a}& &= \vint\brk{(\rho+\ploc)\gamma^2 K\cdt l_a}
        +\aint \brk{(\sigma_E-\sigma)\gamma^2 K\cdt l_a},\\
    S &= Q_S& &= \vint\brk{\gamma s}+\aint\brk{\gamma \sigma_S},\\
    R_i &= Q_{R_i}& &= \vint\brk{\gamma \rl_i}+\aint\brk{\gamma \sigma_{R_i}}.\\
  \end{aligned}
\end{equation}
%

From these quantities, we can compute overall angular velocities $\Omega_a$, temperature $T$ and chemical potentials $\mg_i$ thermodynamically
%
\begin{equation}\label{chpotdef:eq}
  \dr E = \Omega_a \,\dr L_a + T \,\dr S + \mg_i \,\dr R_i.
\end{equation}
%
\emph{A priori}, it may not seems that these quantities have to be the same as $\Omega_a$, $T$ and $\mg_i$ from \eqref{rigidrot:eq} and \eqref{rotsol:eq}. However, we can show that they are the same by checking that \eqref{chpotdef:eq} holds with $\Omega_a$, $T$ and $\mg_i$ taken from \eqref{rigidrot:eq} and \eqref{rotsol:eq}. In practice, it is easier to verify the equivalent statement
%
\begin{equation}\label{chpotcheck:eq}
  \dr(E -\Omega_a L_a - T S - \mg_i R_i) = - L_a \,\dr \Omega_a - S \,\dr T - R_i \,\dr \mg_i.
\end{equation}
%

First, making use of \eqref{inttherm:eq} and \eqref{surftherm:eq}, we see that
%
\begin{equation}\label{thermpot:eq}
  \Phi \equiv E -\Omega_a L_a - T S - \mg_i R_i = -Q_K - T Q_S - \mg_i Q_{R_i}
   = - \vint\,\ploc + \aint\,\sigma.
\end{equation}
%

Let's split $\dr\Phi$ into two pieces, the contribution from the change in $(\ploc,\sigma)$ and the contribution from  the change in the region occupied by the fluid:
%
\begin{equation*}
  \dr\Phi = \dr\Phi_{\ploc,\sigma} + \dr\Phi_f
\end{equation*}
%

Consider an infinitesimal change of $\Omega_a$, $T$ and $\mg_i$. We have
%
\begin{equation*}
\begin{split}
  \dr \ploc &= s\,\dr(\gamma T) + r_i\,\dr(\gamma\mg_i) = \frac{\rho+\ploc}{\gamma}\,\dr\gamma + \gamma s\,\dr T +\gamma  \rl_i\,\dr\mg_i,\\
  \dr \sigma &= -\sigma_S\,\dr(\gamma T) - \sigma_{R_i}\,\dr(\gamma\mg_i) = -\frac{\sigma_E-\sigma}{\gamma}\,\dr\gamma - \gamma \sigma_S\,\dr T -\gamma \sigma_{R_i}\,\dr\mg_i,\\
  \gamma^{-3}\,\dr\gamma &= K\cdt\dr K = K\cdt l_a \,\dr\Omega_a.
\end{split}
\end{equation*}
%
From this, we see that
%
\begin{equation*}
  \dr\Phi_{\ploc,\sigma} = - L_a \,\dr \Omega_a - S \,\dr T - R_i \,\dr \mg_i.
\end{equation*}
%
This means that we need $\dr\Phi_f=0$ if \eqref{chpotcheck:eq} is to hold.

The first part of $\dr\Phi_f=0$ is
%
\begin{equation*}
  -\dr\brk{\vint\,\ploc}_f = -\dvint\,\ploc,
\end{equation*}
%
where $\Delta\Gamma$ is the region between the new and the old surfaces (with appropriate signs).

The change in the surface area can be written as
%
\begin{equation*}
  \dr\brk{\aint\,\sigma}_f = -\daint\,\sigma \hat{n}\cdt\tilde{n},
\end{equation*}
%
where $\hat{n}$ is a unit normal vector pointing into the initial fluid and out of the final fluid and $\tilde{n}$ is some vector field that is equal to the outward pointing normal at both the initial and final surfaces. By Gauss' theorem, this can be written as
%
\begin{equation*}
  \dr\brk{\aint\,\sigma}_f= \dvint, \nabla\cdt(\sigma\tilde{n}),
\end{equation*}
%
As the region of integration is already infinitesimal, we can replace $\tilde{n}$ with the vector field $n$ described in \eqref{normoffsurf:eq}, as the difference would be infinitesimal, i.e.\ $\nabla\cdt(\sigma\tilde{n}) \to \sigma\Theta + n\cdt\sigma$.
Making use of \eqref{rotsigmagrad:eq}
%
\begin{equation*}
  \\dr\Phi_f = \dvint\brk{\sigma\Theta + (\sigma_E-\sigma)n\cdt a - \ploc}.
\end{equation*}
%
which vanishes due to \eqref{rotsurfbc:eq}.

The thermodynamics of the solution can be summarised by defining a grand partition function
%
\begin{equation}\label{gpf:eq}
  \gpf = \Tr\exp\prn{-\frac{E -\Omega_a L_a  - \mg_i R_i}{T}}.
\end{equation}
%
In the thermodynamic limit,
%
\begin{equation}\label{gpftherm:eq}
  \begin{split}
    -T\ln\gpf &= \Phi = E -\Omega_a L_a - T S - \mg_i R_i, \\
    \dr(T\ln\gpf) &= L_a \,\dr \Omega_a + S \,\dr T + R_i \,\dr \mg_i.
  \end{split}
\end{equation}
%
We have seen that
%
\begin{equation}\label{gpfrot:eq}
  T\ln\gpf = \int_{f>0}\!\!\!\dr V \,\ploc - \int_{f=0}\!\!\!\dr A \,\sigma
\end{equation}
%
and the $\Omega_a$, $T$ and $\mg_i$ are the same as those given by \eqref{rigidrot:eq} and \eqref{rotsol:eq}.



%\section*{Acknowledgements}



%%%%%%%%%%%%%%%%%%%%%%%%%%%%%%%%%%%%%%%%%%%%%%%%%%%%%%%%%%%%%%%%%%%%%%%%%%
\section*{Appendices}
\appendix
%%%%%%%%%%%%%%%%%%%%%%%%%%%%%%%%%%%%%%%%%%%%%%%%%%%%%%%%%%%%%%%%%%%%%%%%%%

\section{Extrinsic curvature}\label{sec:extrinsic}

Suppose we have a timelike surface with unit normal vector $n$ pointing toward us (spacelike surfaces will require some sign differences). The induced metric on the surface is
%
\begin{equation}\label{indmet:eq}
  h_{\mu\nu} = g_{\mu\nu} - n_\mu n_\nu.
\end{equation}
%
The extrinsic curvature is given by \cite{Wald-GeneRela:84}
%
\begin{equation}\label{extrdef:eq}
  \Theta_{\mu\nu} = \half \CL_n h_{\mu\nu} = \nabla_\mu n_\nu.
\end{equation}
%
We have to be a little careful with the last expression. It agrees with the first expression when projected tangent to the surface. The first expression has vanishing components normal to the surface. The normal components of the second expression depend on how we extend $n$ off the surface.

The conventional choice for extending $n$ is as follows: at each point on the surface, construct the geodesic that passes through that point tangent to $n$ and parallel transport $n$ along it. In other words
%
\begin{equation}\label{geodesic:eq}
  n^\mu \nabla_\mu n^\nu = 0.
\end{equation}
%
This ensures that the second expression in \eqref{extrdef:eq} has vanishing components normal to the surface. The other normal component, $n^\nu \nabla_\mu n_\nu$, vanishes due to the normalisation of $n$.

For the surfaces given by $f(x)=0$, considered in \S\ref{sec:surface}, the unit normal on the surface is given by
%
\begin{equation}\label{normonsurf:eq}
  n_\mu = -\frac{\p_\mu f}{\sqrt{\p f \cdt \p f}}.
\end{equation}
%
However, if we used this vector away from the surface, it would not satisfy \eqref{geodesic:eq}. We could still use either expression in \eqref{extrdef:eq} with this vector --- we would just have to project the second one tangent to the surface. Alternatively, we can use
%
\begin{equation}\label{normoffsurf:eq}
  n_\mu = -\frac{\p_\mu f}{(\p f \cdt \p f)^{1/2}}
   +\brk{ \frac{\p^\nu\! f\, \nabla_\nu \p_\mu f}{(\p f \cdt \p f)^{3/2}}
      - \frac{\p_\mu f\, \p^\lambda\! f\, \p^\nu\! f\, \nabla_\lambda \p_\nu f}{(\p f \cdt \p f)^{5/2}} } f
   + \CO(f^2).
\end{equation}
%
The $\CO(f^2)$ terms don't contribute to \eqref{extrdef:eq} or \eqref{geodesic:eq} on the surface. The contribution of the $\CO(f)$ terms on the surface to \eqref{extrdef:eq} are normal to the surface and ensure that $n$ satisfies \eqref{geodesic:eq}.

Either way, on the surface, we get
%
\begin{equation}\label{extrsurf:eq}
  \Theta_{\mu\nu} = -\frac{\nabla_\mu \p_\nu f}{(\p f \cdt \p f)^{1/2}}
    + \frac{\p_\mu f\, \p^\lambda\! f\, \nabla_\lambda \p_\nu f + \p_\nu f\, \p^\lambda\! f\, \nabla_\lambda \p_\mu f}
           {(\p f \cdt \p f)^{3/2}}
    - \frac{\p_\mu f\, \p_\nu f\, \p^\lambda\! f\, \p^\sigma\! f\, \nabla_\lambda \p_\sigma f}
           {(\p f \cdt \p f)^{5/2}}.
\end{equation}
%
As this is perpendicular to $n$, it doesn't matter if we contract its indices with the full metric $g_{\mu\nu}$ or the induced metric $h_{\mu\nu}$. We get
%
\begin{equation}\label{trextrsurf:eq}
  \Theta = \Theta_\mu^\mu = -\frac{\square f}{(\p f \cdt \p f)^{1/2}}
    + \frac{\p^\mu\! f\, \p^\nu\! f\, \nabla_\mu \p_\nu f}
           {(\p f \cdt \p f)^{3/2}}.
\end{equation}
%

\section{Notation}\label{app:notation}

We work in the $(-+++)$ signature. $\mu,\nu$ denote space-time indices, $i,j=1 \ldots c$ label the $c$ different R-charges and $a,b=1 \ldots n$ label the $n$ different angular momenta. The dimensions of the AdS space is denoted by $D$ whereas the spacetime dimensions of its boundary is denoted by $d=D-1$. In this paper we consider fluids on $S^{D-2}\times \R$ or equivalently $S^{d-1}\times \R$. Here we present some relations which are useful in converting between $n$, $D$ and $d$:
%
\begin{equation*}
\begin{split}
 D&= d+1 = 2n+2-(D \mod 2) \\
 d&=D-1 =2n + (d \mod 2) \\
 n&=\left[\frac{D-1}{2}\right]=\left[\frac{d}{2}\right]
\end{split}
\end{equation*}
%
where $[x]$ represents the integer part of a real number $x$.

A summary of the variables used in this paper appears in table \ref{tab:notation}.

\begin{table}
\begin{center}
  \begin{tabular}{||r|l||r|l||}
    \hline
    % after \\: \hline or \cline{col1-col2} \cline{col3-col4} ...
    Symbol & Definition & Symbol & Definition \\
    \hline
    $D$ & Dimension of bulk & $d$ & $D-1$, Dimension of boundary \\
    $G_D$ & Newton Constant in AdS$_D$& $n$ & $[d/2]$,  no.\ of commuting \\
    $V_{d}$ & Volume of $S^{d-1}$, $\frac{2\pi^{d/2}}{\Gamma(d/2)}$ &  & ~~angular momenta \\
    $R_\mathrm{AdS}$ & AdS radius (taken to be unity) & $c$ & no.\ of commuting R-charges \\
    $R_H,r_+$ & Horizon radius & $l_\mathrm{mfp}$ & Mean free path, $\eta /\rho$ \\
    \hline
%    $\floc$ & Plasma free energy & $f$ & Free energy density \\
    $\eloc$ & Fluid energy & $\rho$ & Proper density \\
    $\sloc$ & Fluid entropy & $s$ & Proper entropy density \\
    $\tloc$ & Fluid temperature & $\ploc$ & Pressure \\
    $\rloc_i$ & Fluid R-charge & $\rl_i$ & Proper R-charge density \\
    $\ml_i$ & Fluid chemical potential & $\nu_i$ & $\ml_i/\tloc$ \\
    $\mathcal{V}$ & Volume & $\mathcal{A}$ & Area \\
%    $\Phi$ & Thermodynamic potential \eqref{phidef:eq} & $h$ & $\ploc/\tloc^d$, see \eqref{hdef:eq} \\
%    $h_i$ & $\p h / \p\nu_i$ & $\Gamma$ & Thermodynamic potential \eqref{Gammadef:eq} \\
    \hline
%%break start
%  \end{tabular}
%
%  \begin{tabular}{||r|l||r|l||}
%    \hline
%    Symbol & Definition & Symbol & Definition \\
%    \hline
%%break end
    $T^{\mu\nu}$ & Stress tensor & $J^\mu_S$ & Entropy current \\
    $J^\mu_i$ & R-charge current & $u^\mu$ & $\dr x^\mu/\dr\tau =\gamma(1,\vec{v})$, fluid velocity \\
    $\Omega_a$ & Angular velocities & $\gamma$ & $\prn{1-v^2}^{-1/2}$ \\
    $v^2$ & $\sum_a g_{\phi_a\phi_a}\Omega_a^2$ & $P^{\mu\nu}$ & Projection tensor, $g^{\mu\nu}+u^\mu u^\nu$ \\
    $a^\mu,\vartheta,\sigma^{\mu\nu}$ & see \eqref{fluidtensors:eq} & $\zeta,\eta$ & Bulk, shear viscosity \\
    $q^\mu$ & Heat flux, see \eqref{heatcond:eq} & $\kappa$ & Thermal conductivity \\
    $j^\mu_{i}$ & Diffusion current, see \eqref{fluidtensors:eq} & $D_{ij}$ & Diffusion coefficients \\
    \hline
%%break start
%  \end{tabular}
%
%  \begin{tabular}{||r|l||r|l||}
%    \hline
%    Symbol & Definition & Symbol & Definition \\
%    \hline
%%break end
    $E$ & Total energy \eqref{thermcharge:eq} & $S$ & Total entropy \eqref{thermcharge:eq} \\
    $L_a$ & Angular momenta \eqref{thermcharge:eq} & $R_i$ & Total R-charges \eqref{thermcharge:eq} \\
    $\Omega_a$ & Angular velocities \eqref{chpotdef:eq} & $T$ & Overall temperature \eqref{chpotdef:eq} \\
    $\mg_i$ & Overall chemical potentials \eqref{chpotdef:eq} & $\gpf$ & Partition function \eqref{gpf:eq} \\
    $\Phi$ & Grand thermodynamic potential \eqref{thermpot:eq} & $\Gamma$ & region occupied by fluid \\
    \hline
%%break start
%  \end{tabular}
%
%  \begin{tabular}{||r|l||r|l||}
%    \hline
%    Symbol & Definition & Symbol & Definition \\
%    \hline
% %%break end
    $\Theta^{\mu\nu}$ & Extrinsic curvature \eqref{extrdef:eq} & $\Theta$ & $\Theta^\mu_\mu$ \\
    $n^\mu$ & Unit normal to surface & $h^{\mu\nu}$ & Induced metric of surface \\
    $\sigma$ & Surface tension & $f(x)$ & Surface at $f(x)=0$ \\
    $\sigma_E$ & Surface energy density & $\sigma_S$ & Surface entropy density \\
    $\sigma_{R_i}$ & Surface charge density & & \\
%    $\tc$ & Integration constant \eqref{solution:eq} & $\kappa_i$ & Thermodynamic parameters \eqref{bbeosparam:eq} \\
%    $m,q,s_i,H_i$ & Black hole parameters & $X,Y,Z$ & $1/(1+\kappa_i)$ \eqref{xyzkappa:eq} \\
    \hline
   %break start
  \end{tabular}
\end{center}
\caption{Summary of variables used. Numbers in parentheses refer to the equation where it is defined \label{tab:notation}}
\end{table}




%%%%%%%%%%%%%%%%%%%%%%%%%%%%%%%%%%%%%%%%%%%%%%%%%%%%%%%%%%%%%%%%%%%%%%%%%%

\bibliographystyle{utcaps_sl}
\bibliography{qft,string,adscft,gr,maths}
%\bibliography{equilibrium_notes-minimal}

\end{document}

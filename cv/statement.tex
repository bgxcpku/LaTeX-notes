% -*- TeX -*- -*- US -*-
% ----------------------------------------------------------------
% arXiv Paper ************************************************
%
% Subhaneil Lahiri's template
%
% Before submitting:
%    Comment out hyperref
%    Comment out showkeys
%    Replace \newcommand{\mlim}[2]{{\stackrel{\scriptstyle #1}{#2}}}
\newcommand{\ra}{\rightarrow}
\newcommand{\lr}{\leftrightarrow}
\newcommand{\cdt}{\!\cdot\!}
\newcommand{\vp}{\vspace{0.5cm}}
\newcommand{\degs}{^\circ}
%
%e.g., i.e. with normal spaces
\newcommand{\eg}{e.g.\ }
\newcommand{\ie}{i.e.\ }
\newcommand{\cf}{cf.\ }
\newcommand{\etc}{etc.\ }
%
% indices
\newcommand{\up}[1]{\mbox{}^{#1}}
\newcommand{\dn}[1]{\mbox{}_{#1}}
\newcommand{\rp}[1]{^{(#1)}}
\newcommand{\lp}[1]{_{(#1)}}
%
% brackets etc.
\newcommand{\prn}[1]{\left ( #1 \right )}
\newcommand{\brc}[1]{\left\{ #1 \right\}}
\newcommand{\brk}[1]{\left [ #1 \right ]}
\newcommand{\abs}[1]{\left\lvert #1 \right\rvert}
\newcommand{\nrm}[1]{\left\lVert #1 \right\rVert}
\newcommand{\av}[1]{\left\langle #1 \right\rangle}
%
% QM Dirac notation
\newcommand{\bra}[1]{\left\langle #1 \right \rvert}
\newcommand{\ket}[1]{\left \lvert #1 \right\rangle}
\newcommand{\braket}[2]{\left\langle #1 \midddle | #2 \right\rangle}
\newcommand{\bracket}[3]{\left\langle #1 \middle | #2 \middle | #3 \right\rangle}
%
% Derivatives, etc. First argument is optional.
\newcommand{\diff}[3][\rule{0mm}{0mm}]{\frac{\mathrm{d}^{#1} #2}{\mathrm{d}{#3}^{#1}}}
\newcommand{\pdiff}[3][\rule{0mm}{0mm}]{\frac{\partial^{#1} #2}{\partial {#3}^{#1}}}
\newcommand{\pdiffc}[3][\rule{0mm}{0mm}]{\left (\frac{\partial #2}{\partial {#3}}\right )_{\!\!#1}}
\newcommand{\pdl}[1][\rule{0mm}{0mm}]{\overleftarrow{\partial}_{#1}}
\newcommand{\pdr}[1][\rule{0mm}{0mm}]{\overrightarrow{\partial}_{#1}}
\newcommand{\pdlr}[1][\rule{0mm}{0mm}]{\overleftrightarrow{\partial_{#1}}}
\newcommand{\fdf}[2]{\frac{\delta #1}{\delta #2}}
\newcommand{\intd}[1]{\int\!\dr #1\,}
%
% Un-italicised letters
\newcommand{\dr}{\mathrm{d}}
\newcommand{\e}{\mathrm{e}}
\newcommand{\ir}{\mathrm{i}}
\DeclareMathOperator{\tr}{tr}
\DeclareMathOperator{\Tr}{Tr}
\DeclareMathOperator{\Det}{Det}
%
% The default \Im and \Re look crap
\renewcommand{\Im}{\operatorname{\mathfrak{Im}}}
\renewcommand{\Re}{\operatorname{\mathfrak{Re}}}
%
% Referencing sections, figures, etc
\newcommand{\sref}[1]{\S\ref{#1}}
\newcommand{\cref}[1]{Ch.\ref{#1}}
\newcommand{\Cref}[1]{Ch.\ref{#1}}
\newcommand{\fref}[1]{fig.\ref{#1}}
\newcommand{\Fref}[1]{Fig.\ref{#1}}
\newcommand{\tref}[1]{tab.\ref{#1}}
\newcommand{\Tref}[1]{Tab.\ref{#1}}
%
\newcommand{\nn}{\nonumber}
%
% Put the preprint numbers in the top right corner of the page.
% Use after \maketitle.
% First argument: How high it needs to be raised,
% Second argument: Width of the box,
% Third argument: The preprint numbers.
\newcommand{\preprintno}[3]{\hfill\raisebox{#1}[0cm][0cm]{
\begin{minipage}[t]{#2}\begin{flushright} #3 \end{flushright}\end{minipage}}
\vspace*{-\baselinestretch\baselineskip}}
%
% If you have changed the line spacing, e.g. with \renewcommand{\baselinestretch}{1.5},
% the command \sgap produces a line break with the normal spacing.
\newlength{\lingap}
\setlength{\lingap}{\baselinestretch\baselineskip}
\addtolength{\lingap}{-\baselineskip}
\newcommand{\sgap}{\\[-\lingap]}
 with its contents
%       or include mydefs.tex in zip/tar file
%    Replace %
\newcommand{\CD}{\mathcal{D}}
\newcommand{\CE}{\mathcal{E}}
\newcommand{\CG}{\mathcal{G}}
\newcommand{\CH}{\mathcal{H}}
\newcommand{\CK}{\mathcal{K}}
\newcommand{\CO}{\mathcal{O}}
\newcommand{\CL}{\mathcal{L}}
\newcommand{\CM}{\mathcal{M}}
\newcommand{\CN}{\mathcal{N}}
\newcommand{\CV}{\mathcal{V}}
\newcommand{\CZ}{\mathcal{Z}}
%
\newcommand{\dM}{\mathfrak{M}}
\newcommand{\dmd}{\mathfrak{d}}
\newcommand{\dmD}{\mathfrak{D}}
%
\newcommand{\R}{\mathbb{R}}
\newcommand{\C}{\mathbb{C}}
\newcommand{\CP}{\mathbb{CP}}
\newcommand{\Z}{\mathbb{Z}}
%
\newcommand{\ad}{{\dot{\alpha}}}
\newcommand{\bd}{{\dot{\beta}}}
\newcommand{\gd}{{\dot{\gamma}}}
\newcommand{\dd}{{\dot{\delta}}}
\newcommand{\ed}{{\dot{\epsilon}}}
%
\newcommand{\bs}{\overline{\sigma}}
\newcommand{\br}{\overline{\rho}}
\newcommand{\bpsi}{\overline{\psi}}
\newcommand{\bchi}{\overline{\chi}}
\newcommand{\bPsi}{\overline{\Psi}}
\newcommand{\bQ}{\overline{Q}}
\newcommand{\bS}{\overline{S}}
\newcommand{\bJ}{\overline{J}}
\newcommand{\zb}{{\bar z}}
\newcommand{\wb}{{\overline w}}
\newcommand{\cb}{{\bar c}}
\newcommand{\ab}{{\bar a}}
\newcommand{\bb}{{\bar b}}
\newcommand{\bp}{{\bar\partial}}
%
\newcommand{\p}{\partial}
\newcommand{\apm}{{\alpha^{\prime}}}
\newcommand{\adg}{a^\dagger}
\newcommand{\psq}{^{\prime\,2}}
\newcommand{\ppsq}{^{\prime\prime\,2}}
\newcommand{\half}{\frac{1}{2}}
%
 with its contents
%       or include newsymb.tex in zip/tar file
%    Put this file, the .bbl file, any picture or
%       other additional files and natbib.sty
%       file in a zip/tar file
%
% **** -----------------------------------------------------------
\documentclass[11pt]{article}
% Preamble:
\renewcommand{\refname}{\large References}
\usepackage{a4wide}
\usepackage[centertags]{amsmath}
\usepackage{amssymb}
%\usepackage{amsthm}
\usepackage[sort&compress,numbers]{natbib}
%\usepackage{citeB}
\usepackage{ifpdf}
%\usepackage{graphicx}
%\usepackage{graphics} for finding documentation only
%\usepackage{xcolor}
%\usepackage{pgf}
\ifpdf
\usepackage[pdftex,bookmarks]{hyperref}
\else
\usepackage[hypertex]{hyperref}
%\DeclareGraphicsRule{.png}{eps}{.bb}{}
\fi
%
% ----------------------------------------------------------------
\vfuzz2pt % Don't report over-full v-boxes if over-edge is small
\hfuzz2pt % Don't report over-full h-boxes if over-edge is small
%\numberwithin{equation}{section}
%\renewcommand{\baselinestretch}{1.5}
% ----------------------------------------------------------------
% New commands etc.
\newcommand{\mlim}[2]{{\stackrel{\scriptstyle #1}{#2}}}
\newcommand{\ra}{\rightarrow}
\newcommand{\lr}{\leftrightarrow}
\newcommand{\cdt}{\!\cdot\!}
\newcommand{\vp}{\vspace{0.5cm}}
\newcommand{\degs}{^\circ}
%
%e.g., i.e. with normal spaces
\newcommand{\eg}{e.g.\ }
\newcommand{\ie}{i.e.\ }
\newcommand{\cf}{cf.\ }
\newcommand{\etc}{etc.\ }
%
% indices
\newcommand{\up}[1]{\mbox{}^{#1}}
\newcommand{\dn}[1]{\mbox{}_{#1}}
\newcommand{\rp}[1]{^{(#1)}}
\newcommand{\lp}[1]{_{(#1)}}
%
% brackets etc.
\newcommand{\prn}[1]{\left ( #1 \right )}
\newcommand{\brc}[1]{\left\{ #1 \right\}}
\newcommand{\brk}[1]{\left [ #1 \right ]}
\newcommand{\abs}[1]{\left\lvert #1 \right\rvert}
\newcommand{\nrm}[1]{\left\lVert #1 \right\rVert}
\newcommand{\av}[1]{\left\langle #1 \right\rangle}
%
% QM Dirac notation
\newcommand{\bra}[1]{\left\langle #1 \right \rvert}
\newcommand{\ket}[1]{\left \lvert #1 \right\rangle}
\newcommand{\braket}[2]{\left\langle #1 \midddle | #2 \right\rangle}
\newcommand{\bracket}[3]{\left\langle #1 \middle | #2 \middle | #3 \right\rangle}
%
% Derivatives, etc. First argument is optional.
\newcommand{\diff}[3][\rule{0mm}{0mm}]{\frac{\mathrm{d}^{#1} #2}{\mathrm{d}{#3}^{#1}}}
\newcommand{\pdiff}[3][\rule{0mm}{0mm}]{\frac{\partial^{#1} #2}{\partial {#3}^{#1}}}
\newcommand{\pdiffc}[3][\rule{0mm}{0mm}]{\left (\frac{\partial #2}{\partial {#3}}\right )_{\!\!#1}}
\newcommand{\pdl}[1][\rule{0mm}{0mm}]{\overleftarrow{\partial}_{#1}}
\newcommand{\pdr}[1][\rule{0mm}{0mm}]{\overrightarrow{\partial}_{#1}}
\newcommand{\pdlr}[1][\rule{0mm}{0mm}]{\overleftrightarrow{\partial_{#1}}}
\newcommand{\fdf}[2]{\frac{\delta #1}{\delta #2}}
\newcommand{\intd}[1]{\int\!\dr #1\,}
%
% Un-italicised letters
\newcommand{\dr}{\mathrm{d}}
\newcommand{\e}{\mathrm{e}}
\newcommand{\ir}{\mathrm{i}}
\DeclareMathOperator{\tr}{tr}
\DeclareMathOperator{\Tr}{Tr}
\DeclareMathOperator{\Det}{Det}
%
% The default \Im and \Re look crap
\renewcommand{\Im}{\operatorname{\mathfrak{Im}}}
\renewcommand{\Re}{\operatorname{\mathfrak{Re}}}
%
% Referencing sections, figures, etc
\newcommand{\sref}[1]{\S\ref{#1}}
\newcommand{\cref}[1]{Ch.\ref{#1}}
\newcommand{\Cref}[1]{Ch.\ref{#1}}
\newcommand{\fref}[1]{fig.\ref{#1}}
\newcommand{\Fref}[1]{Fig.\ref{#1}}
\newcommand{\tref}[1]{tab.\ref{#1}}
\newcommand{\Tref}[1]{Tab.\ref{#1}}
%
\newcommand{\nn}{\nonumber}
%
% Put the preprint numbers in the top right corner of the page.
% Use after \maketitle.
% First argument: How high it needs to be raised,
% Second argument: Width of the box,
% Third argument: The preprint numbers.
\newcommand{\preprintno}[3]{\hfill\raisebox{#1}[0cm][0cm]{
\begin{minipage}[t]{#2}\begin{flushright} #3 \end{flushright}\end{minipage}}
\vspace*{-\baselinestretch\baselineskip}}
%
% If you have changed the line spacing, e.g. with \renewcommand{\baselinestretch}{1.5},
% the command \sgap produces a line break with the normal spacing.
\newlength{\lingap}
\setlength{\lingap}{\baselinestretch\baselineskip}
\addtolength{\lingap}{-\baselineskip}
\newcommand{\sgap}{\\[-\lingap]}

%
\newcommand{\CD}{\mathcal{D}}
\newcommand{\CE}{\mathcal{E}}
\newcommand{\CG}{\mathcal{G}}
\newcommand{\CH}{\mathcal{H}}
\newcommand{\CK}{\mathcal{K}}
\newcommand{\CO}{\mathcal{O}}
\newcommand{\CL}{\mathcal{L}}
\newcommand{\CM}{\mathcal{M}}
\newcommand{\CN}{\mathcal{N}}
\newcommand{\CV}{\mathcal{V}}
\newcommand{\CZ}{\mathcal{Z}}
%
\newcommand{\dM}{\mathfrak{M}}
\newcommand{\dmd}{\mathfrak{d}}
\newcommand{\dmD}{\mathfrak{D}}
%
\newcommand{\R}{\mathbb{R}}
\newcommand{\C}{\mathbb{C}}
\newcommand{\CP}{\mathbb{CP}}
\newcommand{\Z}{\mathbb{Z}}
%
\newcommand{\ad}{{\dot{\alpha}}}
\newcommand{\bd}{{\dot{\beta}}}
\newcommand{\gd}{{\dot{\gamma}}}
\newcommand{\dd}{{\dot{\delta}}}
\newcommand{\ed}{{\dot{\epsilon}}}
%
\newcommand{\bs}{\overline{\sigma}}
\newcommand{\br}{\overline{\rho}}
\newcommand{\bpsi}{\overline{\psi}}
\newcommand{\bchi}{\overline{\chi}}
\newcommand{\bPsi}{\overline{\Psi}}
\newcommand{\bQ}{\overline{Q}}
\newcommand{\bS}{\overline{S}}
\newcommand{\bJ}{\overline{J}}
\newcommand{\zb}{{\bar z}}
\newcommand{\wb}{{\overline w}}
\newcommand{\cb}{{\bar c}}
\newcommand{\ab}{{\bar a}}
\newcommand{\bb}{{\bar b}}
\newcommand{\bp}{{\bar\partial}}
%
\newcommand{\p}{\partial}
\newcommand{\apm}{{\alpha^{\prime}}}
\newcommand{\adg}{a^\dagger}
\newcommand{\psq}{^{\prime\,2}}
\newcommand{\ppsq}{^{\prime\prime\,2}}
\newcommand{\half}{\frac{1}{2}}
%

%
%%%%%%%%%%%%%%%%%%%%%%%%%%%%%%%%%%%%%%%%%%%%%%%%%%%%%%%%%%%%%%%%%%%%%%%%%%
% Title info:
\title{Research statement}
%
% Author List:
%
\author{Subhaneil Lahiri}

\begin{document}

\maketitle


%%%%%%%%%%%%%%%%%%%%%%%%%%%%%%%%%%%%%%%%%%%%%%%%%%%%%%%%%%%%%%%%%%%%%%%%%%



%%%%%%%%%%%%%%%%%%%%%%%%%%%%%%%%%%%%%%%%%%%%%%%%%%%%%%%%%%%%%%%%%%%%%%%%%%
% Beginning of Article:
%%%%%%%%%%%%%%%%%%%%%%%%%%%%%%%%%%%%%%%%%%%%%%%%%%%%%%%%%%%%%%%%%%%%%%%%%%


I am interested in research in the field of computational neuroscience. I would like to investigate how networks of neurons are able to process information and perform the tasks that underlie animal behavior. My background in theoretical physics leaves me well trained in the analytic and quantitative approach to studying nature. Recently, I have been working in behavioral neuroscience, gaining experience in how to think about biological systems and working with experimental data.

\subsection*{Graduate research (theoretical physics)}

My graduate work in string theory was based on the much celebrated AdS/CFT correspondence that relates certain theories of gravity to certain lower dimensional non-gravitational theories, particularly in the use of the non-gravitational theory to study black holes. My preferred approach is to attack the larger, more complex problem by recasting it in a simpler model system that captures many of the essential features of the original problem whilst making it tractable without overly elaborate techniques, allowing one to develop some intuition.

My PhD thesis was devoted to the regime where the non-gravitational theory is well approximated by fluid mechanics. Most people use this by studying the black holes to calculate properties of the fluid. In \cite{Lahiri:2007ae,Bhattacharyya:2007vs,Bhattacharya:2009gm}, we turned this around and used the fluid to study the black holes. This approach uses a branch of physics that is several centuries old, in which we have developed a great deal of intuition from theory as well as everyday experience, to provide insight into aspects of black hole physics that are difficult to study directly.

The earlier part of my graduate work was concerned with the thermodynamics of black holes with extra symmetries that could render comparisons between the two theories tractable. In \cite{Biswas:2006tj}, we studied a sector with greater symmetry and were able to demonstrate that the detailed thermodynamics of the two theories match exactly, suggesting that the this approach could work when applied to the black hole.

\subsection*{Postdoctoral research (neurobiology)}

My postdoctoral work has focused on the behavior of the \emph{Drosophila} larva. This is a particularly convenient model organism as it has a relatively small number of neurons, making it simpler than larger animals, there are many genetic tools to manipulate its nervous system and its body is transparent, allowing the use of optogenetics. It differs from other model organisms, such as \emph{C.\ elegans} in having a central nervous system and spiking neurons, making it more similar to larger, more complex organisms.

Navigation is a convenient type of behavior to study. The stimuli of interest, as well as the results of this behavior, can be characterized and controlled or measured effectively. Ultimately, one would like to characterize every part of the animal responsible for this, from the sensory neurons that receive the input, to the central nervous system that process the information and chooses the motion, to the muscles that produce the motor output.

In \cite{Lahiri2011}, we used fluorescent imaging of the muscle system in freely moving larvae to study the motor output in detail, using semi-automated image analysis to quantify patterns of muscle contraction. We showed that the movements of the animals during navigation are composed of just two motor programs. To control navigation, the larval brain need only choose when, and how much, to use each program.

\subsection*{Future directions}

I would like to pursue research in computational neuroscience. I find the way that internal representations of the outside world are encoded in a series of action potentials intriguing. The fact that we seem to perceive a definite picture of our surroundings despite the sparsity of this description is fascinating.

I am particularly interested in questions regarding how the brain is able to deal with the limited and noisy nature of its inputs, for example the extent to which the properties of the outside world can be inferred from this, the mechanisms for this inference, and how this affects decision making and control of movement.




%\section*{Acknowledgements}



%%%%%%%%%%%%%%%%%%%%%%%%%%%%%%%%%%%%%%%%%%%%%%%%%%%%%%%%%%%%%%%%%%%%%%%%%%
%\section*{Appendices}
%\appendix
%%%%%%%%%%%%%%%%%%%%%%%%%%%%%%%%%%%%%%%%%%%%%%%%%%%%%%%%%%%%%%%%%%%%%%%%%%





%%%%%%%%%%%%%%%%%%%%%%%%%%%%%%%%%%%%%%%%%%%%%%%%%%%%%%%%%%%%%%%%%%%%%%%%%%

\bibliographystyle{utcaps_sl}
\bibliography{qft,string,adscft,gr,maths,larva}

\end{document}

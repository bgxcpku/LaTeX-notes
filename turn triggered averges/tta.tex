% -*- TeX -*- -*- UK -*-
% ----------------------------------------------------------------
% arXiv Paper ************************************************
%
% Subhaneil Lahiri's template
%
% Before submitting:
%    Comment out hyperref
%    Comment out showkeys
%    Replace \newcommand{\mlim}[2]{{\stackrel{\scriptstyle #1}{#2}}}
\newcommand{\ra}{\rightarrow}
\newcommand{\lr}{\leftrightarrow}
\newcommand{\cdt}{\!\cdot\!}
\newcommand{\vp}{\vspace{0.5cm}}
\newcommand{\degs}{^\circ}
%
%e.g., i.e. with normal spaces
\newcommand{\eg}{e.g.\ }
\newcommand{\ie}{i.e.\ }
\newcommand{\cf}{cf.\ }
\newcommand{\etc}{etc.\ }
%
% indices
\newcommand{\up}[1]{\mbox{}^{#1}}
\newcommand{\dn}[1]{\mbox{}_{#1}}
\newcommand{\rp}[1]{^{(#1)}}
\newcommand{\lp}[1]{_{(#1)}}
%
% brackets etc.
\newcommand{\prn}[1]{\left ( #1 \right )}
\newcommand{\brc}[1]{\left\{ #1 \right\}}
\newcommand{\brk}[1]{\left [ #1 \right ]}
\newcommand{\abs}[1]{\left\lvert #1 \right\rvert}
\newcommand{\nrm}[1]{\left\lVert #1 \right\rVert}
\newcommand{\av}[1]{\left\langle #1 \right\rangle}
%
% QM Dirac notation
\newcommand{\bra}[1]{\left\langle #1 \right \rvert}
\newcommand{\ket}[1]{\left \lvert #1 \right\rangle}
\newcommand{\braket}[2]{\left\langle #1 \midddle | #2 \right\rangle}
\newcommand{\bracket}[3]{\left\langle #1 \middle | #2 \middle | #3 \right\rangle}
%
% Derivatives, etc. First argument is optional.
\newcommand{\diff}[3][\rule{0mm}{0mm}]{\frac{\mathrm{d}^{#1} #2}{\mathrm{d}{#3}^{#1}}}
\newcommand{\pdiff}[3][\rule{0mm}{0mm}]{\frac{\partial^{#1} #2}{\partial {#3}^{#1}}}
\newcommand{\pdiffc}[3][\rule{0mm}{0mm}]{\left (\frac{\partial #2}{\partial {#3}}\right )_{\!\!#1}}
\newcommand{\pdl}[1][\rule{0mm}{0mm}]{\overleftarrow{\partial}_{#1}}
\newcommand{\pdr}[1][\rule{0mm}{0mm}]{\overrightarrow{\partial}_{#1}}
\newcommand{\pdlr}[1][\rule{0mm}{0mm}]{\overleftrightarrow{\partial_{#1}}}
\newcommand{\fdf}[2]{\frac{\delta #1}{\delta #2}}
\newcommand{\intd}[1]{\int\!\dr #1\,}
%
% Un-italicised letters
\newcommand{\dr}{\mathrm{d}}
\newcommand{\e}{\mathrm{e}}
\newcommand{\ir}{\mathrm{i}}
\DeclareMathOperator{\tr}{tr}
\DeclareMathOperator{\Tr}{Tr}
\DeclareMathOperator{\Det}{Det}
%
% The default \Im and \Re look crap
\renewcommand{\Im}{\operatorname{\mathfrak{Im}}}
\renewcommand{\Re}{\operatorname{\mathfrak{Re}}}
%
% Referencing sections, figures, etc
\newcommand{\sref}[1]{\S\ref{#1}}
\newcommand{\cref}[1]{Ch.\ref{#1}}
\newcommand{\Cref}[1]{Ch.\ref{#1}}
\newcommand{\fref}[1]{fig.\ref{#1}}
\newcommand{\Fref}[1]{Fig.\ref{#1}}
\newcommand{\tref}[1]{tab.\ref{#1}}
\newcommand{\Tref}[1]{Tab.\ref{#1}}
%
\newcommand{\nn}{\nonumber}
%
% Put the preprint numbers in the top right corner of the page.
% Use after \maketitle.
% First argument: How high it needs to be raised,
% Second argument: Width of the box,
% Third argument: The preprint numbers.
\newcommand{\preprintno}[3]{\hfill\raisebox{#1}[0cm][0cm]{
\begin{minipage}[t]{#2}\begin{flushright} #3 \end{flushright}\end{minipage}}
\vspace*{-\baselinestretch\baselineskip}}
%
% If you have changed the line spacing, e.g. with \renewcommand{\baselinestretch}{1.5},
% the command \sgap produces a line break with the normal spacing.
\newlength{\lingap}
\setlength{\lingap}{\baselinestretch\baselineskip}
\addtolength{\lingap}{-\baselineskip}
\newcommand{\sgap}{\\[-\lingap]}
 with its contents
%       or include mydefs.tex in zip/tar file
%    Replace %
\newcommand{\CD}{\mathcal{D}}
\newcommand{\CE}{\mathcal{E}}
\newcommand{\CG}{\mathcal{G}}
\newcommand{\CH}{\mathcal{H}}
\newcommand{\CK}{\mathcal{K}}
\newcommand{\CO}{\mathcal{O}}
\newcommand{\CL}{\mathcal{L}}
\newcommand{\CM}{\mathcal{M}}
\newcommand{\CN}{\mathcal{N}}
\newcommand{\CV}{\mathcal{V}}
\newcommand{\CZ}{\mathcal{Z}}
%
\newcommand{\dM}{\mathfrak{M}}
\newcommand{\dmd}{\mathfrak{d}}
\newcommand{\dmD}{\mathfrak{D}}
%
\newcommand{\R}{\mathbb{R}}
\newcommand{\C}{\mathbb{C}}
\newcommand{\CP}{\mathbb{CP}}
\newcommand{\Z}{\mathbb{Z}}
%
\newcommand{\ad}{{\dot{\alpha}}}
\newcommand{\bd}{{\dot{\beta}}}
\newcommand{\gd}{{\dot{\gamma}}}
\newcommand{\dd}{{\dot{\delta}}}
\newcommand{\ed}{{\dot{\epsilon}}}
%
\newcommand{\bs}{\overline{\sigma}}
\newcommand{\br}{\overline{\rho}}
\newcommand{\bpsi}{\overline{\psi}}
\newcommand{\bchi}{\overline{\chi}}
\newcommand{\bPsi}{\overline{\Psi}}
\newcommand{\bQ}{\overline{Q}}
\newcommand{\bS}{\overline{S}}
\newcommand{\bJ}{\overline{J}}
\newcommand{\zb}{{\bar z}}
\newcommand{\wb}{{\overline w}}
\newcommand{\cb}{{\bar c}}
\newcommand{\ab}{{\bar a}}
\newcommand{\bb}{{\bar b}}
\newcommand{\bp}{{\bar\partial}}
%
\newcommand{\p}{\partial}
\newcommand{\apm}{{\alpha^{\prime}}}
\newcommand{\adg}{a^\dagger}
\newcommand{\psq}{^{\prime\,2}}
\newcommand{\ppsq}{^{\prime\prime\,2}}
\newcommand{\half}{\frac{1}{2}}
%
 with its contents
%       or include newsymb.tex in zip/tar file
%    Put this file, the .bbl file, any picture or
%       other additional files and natbib.sty
%       file in a zip/tar file
%
% **** -----------------------------------------------------------
\documentclass[12pt]{article}
% Preamble:
\usepackage{a4wide}
\usepackage[centertags]{amsmath}
\usepackage{amssymb}
\usepackage[sort&compress,numbers]{natbib}
%\usepackage{citeB}
\usepackage{ifpdf}
\usepackage{graphicx}
%\usepackage{graphics} for finding documentation only
%\usepackage{xcolor}
%\usepackage{pgf}
\ifpdf
\usepackage[pdftex,bookmarks]{hyperref}
\else
\usepackage[hypertex]{hyperref}
\DeclareGraphicsRule{.png}{eps}{.bb}{}
\fi
%
% >> Only for drafts! <<
\usepackage[notref,notcite]{showkeys}
% ----------------------------------------------------------------
\vfuzz2pt % Don't report over-full v-boxes if over-edge is small
\hfuzz2pt % Don't report over-full h-boxes if over-edge is small
%\numberwithin{equation}{section}
%\renewcommand{\baselinestretch}{1.5}
% ----------------------------------------------------------------
\usepackage{amsthm}
\newtheoremstyle{sldefinition}%
  {3pt}%space above
  {3pt}%space below
  {}%body font
  {}%indent amount
  {\bfseries}%theorem head font
  {}%theorem head punctuation
  {\newline}%space after head
  {\thmname{#1}\thmnumber{ #2}:{\bfseries\thmnote{ #3}}}%head spec
\newtheoremstyle{slplain}%
  {3pt}%space above
  {3pt}%space below
  {}%body font
  {}%indent amount
  {\bfseries}%theorem head font
  {}%theorem head punctuation
  {\newline}%space after head
  {\thmname{#1}\thmnumber{ #2}{\bfseries\thmnote{ (#3)}}:}%head spec
%
\theoremstyle{slplain}
\newtheorem{thm}{Theorem}
%
\theoremstyle{sldefinition}
\newtheorem{defn}{Definition}
%
\theoremstyle{remark}
\newtheorem*{notn}{Notation}
\newtheorem*{rem}{Remark}
% ----------------------------------------------------------------
% New commands etc.
\newcommand{\mlim}[2]{{\stackrel{\scriptstyle #1}{#2}}}
\newcommand{\ra}{\rightarrow}
\newcommand{\lr}{\leftrightarrow}
\newcommand{\cdt}{\!\cdot\!}
\newcommand{\vp}{\vspace{0.5cm}}
\newcommand{\degs}{^\circ}
%
%e.g., i.e. with normal spaces
\newcommand{\eg}{e.g.\ }
\newcommand{\ie}{i.e.\ }
\newcommand{\cf}{cf.\ }
\newcommand{\etc}{etc.\ }
%
% indices
\newcommand{\up}[1]{\mbox{}^{#1}}
\newcommand{\dn}[1]{\mbox{}_{#1}}
\newcommand{\rp}[1]{^{(#1)}}
\newcommand{\lp}[1]{_{(#1)}}
%
% brackets etc.
\newcommand{\prn}[1]{\left ( #1 \right )}
\newcommand{\brc}[1]{\left\{ #1 \right\}}
\newcommand{\brk}[1]{\left [ #1 \right ]}
\newcommand{\abs}[1]{\left\lvert #1 \right\rvert}
\newcommand{\nrm}[1]{\left\lVert #1 \right\rVert}
\newcommand{\av}[1]{\left\langle #1 \right\rangle}
%
% QM Dirac notation
\newcommand{\bra}[1]{\left\langle #1 \right \rvert}
\newcommand{\ket}[1]{\left \lvert #1 \right\rangle}
\newcommand{\braket}[2]{\left\langle #1 \midddle | #2 \right\rangle}
\newcommand{\bracket}[3]{\left\langle #1 \middle | #2 \middle | #3 \right\rangle}
%
% Derivatives, etc. First argument is optional.
\newcommand{\diff}[3][\rule{0mm}{0mm}]{\frac{\mathrm{d}^{#1} #2}{\mathrm{d}{#3}^{#1}}}
\newcommand{\pdiff}[3][\rule{0mm}{0mm}]{\frac{\partial^{#1} #2}{\partial {#3}^{#1}}}
\newcommand{\pdiffc}[3][\rule{0mm}{0mm}]{\left (\frac{\partial #2}{\partial {#3}}\right )_{\!\!#1}}
\newcommand{\pdl}[1][\rule{0mm}{0mm}]{\overleftarrow{\partial}_{#1}}
\newcommand{\pdr}[1][\rule{0mm}{0mm}]{\overrightarrow{\partial}_{#1}}
\newcommand{\pdlr}[1][\rule{0mm}{0mm}]{\overleftrightarrow{\partial_{#1}}}
\newcommand{\fdf}[2]{\frac{\delta #1}{\delta #2}}
\newcommand{\intd}[1]{\int\!\dr #1\,}
%
% Un-italicised letters
\newcommand{\dr}{\mathrm{d}}
\newcommand{\e}{\mathrm{e}}
\newcommand{\ir}{\mathrm{i}}
\DeclareMathOperator{\tr}{tr}
\DeclareMathOperator{\Tr}{Tr}
\DeclareMathOperator{\Det}{Det}
%
% The default \Im and \Re look crap
\renewcommand{\Im}{\operatorname{\mathfrak{Im}}}
\renewcommand{\Re}{\operatorname{\mathfrak{Re}}}
%
% Referencing sections, figures, etc
\newcommand{\sref}[1]{\S\ref{#1}}
\newcommand{\cref}[1]{Ch.\ref{#1}}
\newcommand{\Cref}[1]{Ch.\ref{#1}}
\newcommand{\fref}[1]{fig.\ref{#1}}
\newcommand{\Fref}[1]{Fig.\ref{#1}}
\newcommand{\tref}[1]{tab.\ref{#1}}
\newcommand{\Tref}[1]{Tab.\ref{#1}}
%
\newcommand{\nn}{\nonumber}
%
% Put the preprint numbers in the top right corner of the page.
% Use after \maketitle.
% First argument: How high it needs to be raised,
% Second argument: Width of the box,
% Third argument: The preprint numbers.
\newcommand{\preprintno}[3]{\hfill\raisebox{#1}[0cm][0cm]{
\begin{minipage}[t]{#2}\begin{flushright} #3 \end{flushright}\end{minipage}}
\vspace*{-\baselinestretch\baselineskip}}
%
% If you have changed the line spacing, e.g. with \renewcommand{\baselinestretch}{1.5},
% the command \sgap produces a line break with the normal spacing.
\newlength{\lingap}
\setlength{\lingap}{\baselinestretch\baselineskip}
\addtolength{\lingap}{-\baselineskip}
\newcommand{\sgap}{\\[-\lingap]}

%
\newcommand{\CD}{\mathcal{D}}
\newcommand{\CE}{\mathcal{E}}
\newcommand{\CG}{\mathcal{G}}
\newcommand{\CH}{\mathcal{H}}
\newcommand{\CK}{\mathcal{K}}
\newcommand{\CO}{\mathcal{O}}
\newcommand{\CL}{\mathcal{L}}
\newcommand{\CM}{\mathcal{M}}
\newcommand{\CN}{\mathcal{N}}
\newcommand{\CV}{\mathcal{V}}
\newcommand{\CZ}{\mathcal{Z}}
%
\newcommand{\dM}{\mathfrak{M}}
\newcommand{\dmd}{\mathfrak{d}}
\newcommand{\dmD}{\mathfrak{D}}
%
\newcommand{\R}{\mathbb{R}}
\newcommand{\C}{\mathbb{C}}
\newcommand{\CP}{\mathbb{CP}}
\newcommand{\Z}{\mathbb{Z}}
%
\newcommand{\ad}{{\dot{\alpha}}}
\newcommand{\bd}{{\dot{\beta}}}
\newcommand{\gd}{{\dot{\gamma}}}
\newcommand{\dd}{{\dot{\delta}}}
\newcommand{\ed}{{\dot{\epsilon}}}
%
\newcommand{\bs}{\overline{\sigma}}
\newcommand{\br}{\overline{\rho}}
\newcommand{\bpsi}{\overline{\psi}}
\newcommand{\bchi}{\overline{\chi}}
\newcommand{\bPsi}{\overline{\Psi}}
\newcommand{\bQ}{\overline{Q}}
\newcommand{\bS}{\overline{S}}
\newcommand{\bJ}{\overline{J}}
\newcommand{\zb}{{\bar z}}
\newcommand{\wb}{{\overline w}}
\newcommand{\cb}{{\bar c}}
\newcommand{\ab}{{\bar a}}
\newcommand{\bb}{{\bar b}}
\newcommand{\bp}{{\bar\partial}}
%
\newcommand{\p}{\partial}
\newcommand{\apm}{{\alpha^{\prime}}}
\newcommand{\adg}{a^\dagger}
\newcommand{\psq}{^{\prime\,2}}
\newcommand{\ppsq}{^{\prime\prime\,2}}
\newcommand{\half}{\frac{1}{2}}
%

%
%%%%%%%%%%%%%%%%%%%%%%%%%%%%%%%%%%%%%%%%%%%%%%%%%%%%%%%%%%%%%%%%%%%%%%%%%%
% Title info:
\title{Non-linear, history-dependent response functions}
%
% Author List:
%
\author{Subhaneil Lahiri
\\
%
% Addresses:
%
\small{\emph{Harvard University}}
%
}

\begin{document}

\maketitle

%% Preprint numbers, etc.
%\preprintno{8cm}{6cm}{
%    \texttt{arXiv:yymm.nnnn [hep-th]}
%}

%%%%%%%%%%%%%%%%%%%%%%%%%%%%%%%%%%%%%%%%%%%%%%%%%%%%%%%%%%%%%%%%%%%%%%%%%%


\begin{abstract}
  We look at methods of characterising non-linear, history-dependent responses, particularly the rates of inhomogeneous Poisson processes.
\end{abstract}


%%%%%%%%%%%%%%%%%%%%%%%%%%%%%%%%%%%%%%%%%%%%%%%%%%%%%%%%%%%%%%%%%%%%%%%%%%
% Beginning of Article:
%%%%%%%%%%%%%%%%%%%%%%%%%%%%%%%%%%%%%%%%%%%%%%%%%%%%%%%%%%%%%%%%%%%%%%%%%%

\section{Introduction}\label{sec:intro}

In this note we will look at responses, $r(t)$, that are \emph{functionals} of some stimulus, $s(t)$,
%
\begin{equation}\label{eq:response}
  r(t) = r_t[s].
\end{equation}
%

We will look at types of series expansion: the Volterra series in \sref{sec:volterra} that is analogous to the Taylor expansion for function; and the Wiener series in \sref{sec:wiener} that is analogous to the expansion in Hermite polynomials for functions.

When constructing a set of orthogonal polynomials, one has to choose a weighting function for the integration measure:
%
\begin{equation}\label{eq:functionprod}
  f \cdot g = \int f(t) g(t)\, w(t) \dr t.
\end{equation}
%
For Hermite polynomials, one chooses $w(t)=\e^{-t^2}$, for Legendre polynomials, one chooses $w(t)=\theta(1-t)\theta(1+t)$ and for Laguerre polynomials, one chooses $w(t)=\theta(t)\e^{-t}$.

For orthogonal functionals, one has to choose a measure for the space of functions:
%
\begin{equation}\label{eq:functionalprod}
  F \cdot G = \int F[s] G[s]\, W[s] \CD s.
\end{equation}
%
In analogy with the Hermite polynomials, we will choose
%
\begin{equation}\label{eq:whitenoisemeasure}
  W[s] \propto \exp\prn{-\int \frac{s(t)^2}{2\sigma^2}\,\dr t}.
\end{equation}
%

There is a statistical way of looking at this that is helpful when thinking about how to measure these quantities. In the case of functions, one can normalise $w(t)$ and think of it as a probability density, then we have $f \cdot g = \av{f(t)g(t)}$, where $t$ is a random variable whose probability distribution is given by $w(t)$.

For functionals, we should think of the input, $s(t)$, as being a \emph{stochastic process}. The stochastic process corresponding to \eqref{eq:whitenoisemeasure} is called zero-mean, Gaussian white noise. We will discuss this in more detail in \sref{sec:whitenoise}.

When the response function, $r(t)$, is the rate of an inhomogeneous Poisson process, it can be difficult to measure. One has to repeat the stimulus many times and count the events in various time bins. If one wishes to use small time bins, one has to repeat the stimulus more times to get enough data in each bin.

Instead, there is a simpler method where one compute averages of the \emph{stimulus} at various time shifts before each event. With this method, one never needs to determine $r(t)$. We will discuss this further in \sref{sec:poisson}.


\section{Volterrra series}\label{sec:volterra}

Suppose we have a functional that is also a function of the independent variable, \eg if we have a system that responds to some time varying input, $x(t)$, in a history dependent way (\ie the system has some memory), then the response at time $t$ is a function of $t$ as well as a functional of $x(t')$.
The Volterra series is a way of expanding such a functional in powers of the input. It is analogous to the Taylor series expansion of a function.

\begin{defn}[Volterra functional]
  A Volterra functional of degree $p$ is one of the form:
  %
  \begin{equation}\label{eq:volfunc}
    V^{(p)}_t[x] = \prod_{i=1}^p \int\!\dr t_i\,  v(t_1,\ldots,t_p) \prod_{i=1}^p x(t-t_i).
  \end{equation}
  %
  Without loss of generality, we may assume that the function $v(t_1,\ldots,t_p)$ is symmetric under interchange of any of its arguments
\end{defn}


\begin{defn}[Volterra series]
  A Volterra series of degree $n$ is a sum of Volterra functionals of degree less than or equal to $n$:
  %
  \begin{equation}\label{eq:volser}
    F_t[x] = \sum_{n=0}^n\prod_{i=1}^p \int\!\dr t_i\,  f_p(t_1,\ldots,t_p) \prod_{i=1}^p x(t-t_i).
  \end{equation}
  %
  The term ``Voltera series'', without specification of degree, shall be taken to mean a Volterra series of infinite degree.
\end{defn}

\begin{defn}[Volterra kernel]
  The $p^{\text{th}}$ Volterra kernel is the function $f_p(t_1,\ldots,t_p)$ that appears in \eqref{eq:volser}.
\end{defn}

\begin{thm}[Determination of Volterra kernels]
  Suppose we are given some functional, $F_t[x]$, and we want to express it as a Volterra sreies. The Volterra kernels that we would use are given by
  %
  \begin{equation}\label{eq:volkern}
    f_p(t_1,\ldots,t_p) = \frac{1}{p!}\, \frac{\delta^p F_t[0]}{\delta x(t-t_1) \ldots \delta x(t-t_p)}.
  \end{equation}
  %
\end{thm}
\begin{proof}
  Substitute \eqref{eq:volser} into \eqref{eq:volkern} and see what you get.
\end{proof}

The Volterra expansion is only valid for inputs close to zero if all the functional derivatives in \eqref{eq:volkern} exist, similar to how a Taylor series can only be used for analytic functions. In addition, the Taylor series is only valid inside the radius of convergence, which is equal to the distance to the nearest non-analyticity. This is one of the reasons why orthogonal function expansions are often better than Taylor series: in return for demanding a weaker form of convergence, the class of functions that can be represented is larger.


\begin{rem}[impulse response]
  The experimental analogue of \eqref{eq:volkern} involves measuring the response to a series of impulses. Let
  %
  \begin{equation}\label{eq:imulses}
  \begin{aligned}
    x(t) &= \sum_{i=1}^p \epsilon_i \delta(t-t_i), \\
    F(t;\vec{t},\vec{\epsilon} ) &= F_t[x],
  \end{aligned}
  \end{equation}
  %
  where $\vec{t}$ represents $(t_1,\ldots,t_p)$, etc. Then one can determine the Volterra kernels via
  %
  \begin{equation}\label{eq:impkern}
    f_p(t-t_1,\ldots,t-t_p) = \frac{1}{p!}\, \frac{\p^p F(t;\vec{t},\vec{0})}{\p\epsilon_1\ldots\p\epsilon_p}.
  \end{equation}
  %
  However, one has to perform this experiment several times, varying $\vec{\epsilon}$ and $\vec{t}$, in order to determine one of the Volterra kernels.
\end{rem}


\section{Gaussian white noise}\label{sec:whitenoise}

\begin{defn}[Stationary process]
  A stochastic process, $x(t)$ is said to be stationary if
  %
  \begin{equation}\label{eq:statproc}
    \av{x(t_1)x(t_2) \ldots x(t_n)} = \av{x(t_1+\tau)x(t_2+\tau) \ldots x(t_n+\tau)}
    \qquad \forall n,t_i,\tau
  \end{equation}
  %
\end{defn}

\begin{defn}[Gaussian process]
  A stochastic process is said to be Gaussian if
  %
  \begin{equation}\label{eq:gaussproc}
    \av{\e^{\ir\!\intd{t} [J(t) x(t)]}} = \e^{\ir\!\intd{t} [J(t)\mu(t)] - \intd{t\dr t'} [J(t)J(t')\phi(t,t')]}.
  \end{equation}
  %
  This means that all correlation functions can be expressed in terms of the connected one and two-point functions, \eg
  %
  \begin{equation}\label{eq:gaussmom}
    \begin{aligned}
      \av{x(t_1)} &= \mu(t_1), \\
      \av{x(t_1)x(t_2)}  &= \mu(t_1)\mu(t_2) + \phi(t_1,t_2), \\
      \av{x(t_1)x(t_2)x(t_3)}  &= \mu(t_1)\mu(t_2)\mu(t_3) + \phi(t_1,t_2)\mu(t_3) + \phi(t_1,t_3)\mu(t_2) + \phi(t_2,t_3)\mu(t_1).
    \end{aligned}
  \end{equation}
  %
  If a process is Gaussian and stationary,
  %
  \begin{equation}\label{eq:statgauss}
    \begin{aligned}
      \mu(t) &= \mu, \\
      \phi(t,t')  &= \phi(t-t').
    \end{aligned}
  \end{equation}
  %
  The Fourier transform of $\phi(t)$, $\tilde{\phi}(\omega)$, is called the \textbf{power density spectrum} of $x(t)$.
\end{defn}

\begin{defn}[Gaussian white noise]
  Gaussian white noise of power $\sigma^2$ is a stationary Gaussian process with
  %
  \begin{equation}\label{eq:gaussianwhite}
    \phi(t-t') = \sigma^2\delta(t-t'), \qquad 
    \tilde{\phi}(\omega) = \sigma^2, \qquad
    \av{\e^{\ir\!\intd{t} [J(t) x(t)]}} = \e^{-\sigma^2\!\intd{t} [J(t)]^2}.
  \end{equation}
  %
  \ie the power density spectrum is the same for all frequencies. In practice, one can only have white noise up to some high frequency cut-off.
\end{defn}


\section{Wiener series}\label{sec:wiener}




\section{Inhomogeneous Poisson processes}\label{sec:poisson}





%\section*{Acknowledgements}



%%%%%%%%%%%%%%%%%%%%%%%%%%%%%%%%%%%%%%%%%%%%%%%%%%%%%%%%%%%%%%%%%%%%%%%%%%
%\section*{Appendices}
%\appendix
%%%%%%%%%%%%%%%%%%%%%%%%%%%%%%%%%%%%%%%%%%%%%%%%%%%%%%%%%%%%%%%%%%%%%%%%%%





%%%%%%%%%%%%%%%%%%%%%%%%%%%%%%%%%%%%%%%%%%%%%%%%%%%%%%%%%%%%%%%%%%%%%%%%%%

\bibliographystyle{utcaps_sl}
\bibliography{maths}

\end{document}

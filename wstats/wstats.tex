% -*- TeX -*- -*- UK -*-
% ----------------------------------------------------------------
% arXiv Paper ************************************************
%
% Subhaneil Lahiri's template
%
% Before submitting:
%    Comment out hyperref
%    Comment out showkeys
%    Replace \newcommand{\mlim}[2]{{\stackrel{\scriptstyle #1}{#2}}}
\newcommand{\ra}{\rightarrow}
\newcommand{\lr}{\leftrightarrow}
\newcommand{\cdt}{\!\cdot\!}
\newcommand{\vp}{\vspace{0.5cm}}
\newcommand{\degs}{^\circ}
%
%e.g., i.e. with normal spaces
\newcommand{\eg}{e.g.\ }
\newcommand{\ie}{i.e.\ }
\newcommand{\cf}{cf.\ }
\newcommand{\etc}{etc.\ }
%
% indices
\newcommand{\up}[1]{\mbox{}^{#1}}
\newcommand{\dn}[1]{\mbox{}_{#1}}
\newcommand{\rp}[1]{^{(#1)}}
\newcommand{\lp}[1]{_{(#1)}}
%
% brackets etc.
\newcommand{\prn}[1]{\left ( #1 \right )}
\newcommand{\brc}[1]{\left\{ #1 \right\}}
\newcommand{\brk}[1]{\left [ #1 \right ]}
\newcommand{\abs}[1]{\left\lvert #1 \right\rvert}
\newcommand{\nrm}[1]{\left\lVert #1 \right\rVert}
\newcommand{\av}[1]{\left\langle #1 \right\rangle}
%
% QM Dirac notation
\newcommand{\bra}[1]{\left\langle #1 \right \rvert}
\newcommand{\ket}[1]{\left \lvert #1 \right\rangle}
\newcommand{\braket}[2]{\left\langle #1 \midddle | #2 \right\rangle}
\newcommand{\bracket}[3]{\left\langle #1 \middle | #2 \middle | #3 \right\rangle}
%
% Derivatives, etc. First argument is optional.
\newcommand{\diff}[3][\rule{0mm}{0mm}]{\frac{\mathrm{d}^{#1} #2}{\mathrm{d}{#3}^{#1}}}
\newcommand{\pdiff}[3][\rule{0mm}{0mm}]{\frac{\partial^{#1} #2}{\partial {#3}^{#1}}}
\newcommand{\pdiffc}[3][\rule{0mm}{0mm}]{\left (\frac{\partial #2}{\partial {#3}}\right )_{\!\!#1}}
\newcommand{\pdl}[1][\rule{0mm}{0mm}]{\overleftarrow{\partial}_{#1}}
\newcommand{\pdr}[1][\rule{0mm}{0mm}]{\overrightarrow{\partial}_{#1}}
\newcommand{\pdlr}[1][\rule{0mm}{0mm}]{\overleftrightarrow{\partial_{#1}}}
\newcommand{\fdf}[2]{\frac{\delta #1}{\delta #2}}
\newcommand{\intd}[1]{\int\!\dr #1\,}
%
% Un-italicised letters
\newcommand{\dr}{\mathrm{d}}
\newcommand{\e}{\mathrm{e}}
\newcommand{\ir}{\mathrm{i}}
\DeclareMathOperator{\tr}{tr}
\DeclareMathOperator{\Tr}{Tr}
\DeclareMathOperator{\Det}{Det}
%
% The default \Im and \Re look crap
\renewcommand{\Im}{\operatorname{\mathfrak{Im}}}
\renewcommand{\Re}{\operatorname{\mathfrak{Re}}}
%
% Referencing sections, figures, etc
\newcommand{\sref}[1]{\S\ref{#1}}
\newcommand{\cref}[1]{Ch.\ref{#1}}
\newcommand{\Cref}[1]{Ch.\ref{#1}}
\newcommand{\fref}[1]{fig.\ref{#1}}
\newcommand{\Fref}[1]{Fig.\ref{#1}}
\newcommand{\tref}[1]{tab.\ref{#1}}
\newcommand{\Tref}[1]{Tab.\ref{#1}}
%
\newcommand{\nn}{\nonumber}
%
% Put the preprint numbers in the top right corner of the page.
% Use after \maketitle.
% First argument: How high it needs to be raised,
% Second argument: Width of the box,
% Third argument: The preprint numbers.
\newcommand{\preprintno}[3]{\hfill\raisebox{#1}[0cm][0cm]{
\begin{minipage}[t]{#2}\begin{flushright} #3 \end{flushright}\end{minipage}}
\vspace*{-\baselinestretch\baselineskip}}
%
% If you have changed the line spacing, e.g. with \renewcommand{\baselinestretch}{1.5},
% the command \sgap produces a line break with the normal spacing.
\newlength{\lingap}
\setlength{\lingap}{\baselinestretch\baselineskip}
\addtolength{\lingap}{-\baselineskip}
\newcommand{\sgap}{\\[-\lingap]}
 with its contents
%       or include mydefs.tex in zip/tar file
%    Replace %
\newcommand{\CD}{\mathcal{D}}
\newcommand{\CE}{\mathcal{E}}
\newcommand{\CG}{\mathcal{G}}
\newcommand{\CH}{\mathcal{H}}
\newcommand{\CK}{\mathcal{K}}
\newcommand{\CO}{\mathcal{O}}
\newcommand{\CL}{\mathcal{L}}
\newcommand{\CM}{\mathcal{M}}
\newcommand{\CN}{\mathcal{N}}
\newcommand{\CV}{\mathcal{V}}
\newcommand{\CZ}{\mathcal{Z}}
%
\newcommand{\dM}{\mathfrak{M}}
\newcommand{\dmd}{\mathfrak{d}}
\newcommand{\dmD}{\mathfrak{D}}
%
\newcommand{\R}{\mathbb{R}}
\newcommand{\C}{\mathbb{C}}
\newcommand{\CP}{\mathbb{CP}}
\newcommand{\Z}{\mathbb{Z}}
%
\newcommand{\ad}{{\dot{\alpha}}}
\newcommand{\bd}{{\dot{\beta}}}
\newcommand{\gd}{{\dot{\gamma}}}
\newcommand{\dd}{{\dot{\delta}}}
\newcommand{\ed}{{\dot{\epsilon}}}
%
\newcommand{\bs}{\overline{\sigma}}
\newcommand{\br}{\overline{\rho}}
\newcommand{\bpsi}{\overline{\psi}}
\newcommand{\bchi}{\overline{\chi}}
\newcommand{\bPsi}{\overline{\Psi}}
\newcommand{\bQ}{\overline{Q}}
\newcommand{\bS}{\overline{S}}
\newcommand{\bJ}{\overline{J}}
\newcommand{\zb}{{\bar z}}
\newcommand{\wb}{{\overline w}}
\newcommand{\cb}{{\bar c}}
\newcommand{\ab}{{\bar a}}
\newcommand{\bb}{{\bar b}}
\newcommand{\bp}{{\bar\partial}}
%
\newcommand{\p}{\partial}
\newcommand{\apm}{{\alpha^{\prime}}}
\newcommand{\adg}{a^\dagger}
\newcommand{\psq}{^{\prime\,2}}
\newcommand{\ppsq}{^{\prime\prime\,2}}
\newcommand{\half}{\frac{1}{2}}
%
 with its contents
%       or include newsymb.tex in zip/tar file
%    Put this file, the .bbl file, any picture or
%       other additional files and natbib.sty
%       file in a zip/tar file
%
% **** -----------------------------------------------------------
\documentclass[12pt]{article}
% Preamble:
\usepackage{a4wide}
\usepackage[centertags]{amsmath}
\usepackage{amssymb}
%\usepackage{amsthm}
\usepackage[sort&compress,numbers]{natbib}
%\usepackage{citeB}
\usepackage{ifpdf}
%\usepackage{graphicx}
%\usepackage{graphics} for finding documentation only
%\usepackage{xcolor}
%\usepackage{pgf}
\ifpdf
\usepackage[pdftex,bookmarks]{hyperref}
\else
\usepackage[hypertex]{hyperref}
%\DeclareGraphicsRule{.png}{eps}{.bb}{}
\fi
%
% >> Only for drafts! <<
%\usepackage[notref,notcite]{showkeys}
% ----------------------------------------------------------------
\vfuzz2pt % Don't report over-full v-boxes if over-edge is small
\hfuzz2pt % Don't report over-full h-boxes if over-edge is small
%\numberwithin{equation}{section}
%\renewcommand{\baselinestretch}{1.5}
% ----------------------------------------------------------------
% New commands etc.
\newcommand{\mlim}[2]{{\stackrel{\scriptstyle #1}{#2}}}
\newcommand{\ra}{\rightarrow}
\newcommand{\lr}{\leftrightarrow}
\newcommand{\cdt}{\!\cdot\!}
\newcommand{\vp}{\vspace{0.5cm}}
\newcommand{\degs}{^\circ}
%
%e.g., i.e. with normal spaces
\newcommand{\eg}{e.g.\ }
\newcommand{\ie}{i.e.\ }
\newcommand{\cf}{cf.\ }
\newcommand{\etc}{etc.\ }
%
% indices
\newcommand{\up}[1]{\mbox{}^{#1}}
\newcommand{\dn}[1]{\mbox{}_{#1}}
\newcommand{\rp}[1]{^{(#1)}}
\newcommand{\lp}[1]{_{(#1)}}
%
% brackets etc.
\newcommand{\prn}[1]{\left ( #1 \right )}
\newcommand{\brc}[1]{\left\{ #1 \right\}}
\newcommand{\brk}[1]{\left [ #1 \right ]}
\newcommand{\abs}[1]{\left\lvert #1 \right\rvert}
\newcommand{\nrm}[1]{\left\lVert #1 \right\rVert}
\newcommand{\av}[1]{\left\langle #1 \right\rangle}
%
% QM Dirac notation
\newcommand{\bra}[1]{\left\langle #1 \right \rvert}
\newcommand{\ket}[1]{\left \lvert #1 \right\rangle}
\newcommand{\braket}[2]{\left\langle #1 \midddle | #2 \right\rangle}
\newcommand{\bracket}[3]{\left\langle #1 \middle | #2 \middle | #3 \right\rangle}
%
% Derivatives, etc. First argument is optional.
\newcommand{\diff}[3][\rule{0mm}{0mm}]{\frac{\mathrm{d}^{#1} #2}{\mathrm{d}{#3}^{#1}}}
\newcommand{\pdiff}[3][\rule{0mm}{0mm}]{\frac{\partial^{#1} #2}{\partial {#3}^{#1}}}
\newcommand{\pdiffc}[3][\rule{0mm}{0mm}]{\left (\frac{\partial #2}{\partial {#3}}\right )_{\!\!#1}}
\newcommand{\pdl}[1][\rule{0mm}{0mm}]{\overleftarrow{\partial}_{#1}}
\newcommand{\pdr}[1][\rule{0mm}{0mm}]{\overrightarrow{\partial}_{#1}}
\newcommand{\pdlr}[1][\rule{0mm}{0mm}]{\overleftrightarrow{\partial_{#1}}}
\newcommand{\fdf}[2]{\frac{\delta #1}{\delta #2}}
\newcommand{\intd}[1]{\int\!\dr #1\,}
%
% Un-italicised letters
\newcommand{\dr}{\mathrm{d}}
\newcommand{\e}{\mathrm{e}}
\newcommand{\ir}{\mathrm{i}}
\DeclareMathOperator{\tr}{tr}
\DeclareMathOperator{\Tr}{Tr}
\DeclareMathOperator{\Det}{Det}
%
% The default \Im and \Re look crap
\renewcommand{\Im}{\operatorname{\mathfrak{Im}}}
\renewcommand{\Re}{\operatorname{\mathfrak{Re}}}
%
% Referencing sections, figures, etc
\newcommand{\sref}[1]{\S\ref{#1}}
\newcommand{\cref}[1]{Ch.\ref{#1}}
\newcommand{\Cref}[1]{Ch.\ref{#1}}
\newcommand{\fref}[1]{fig.\ref{#1}}
\newcommand{\Fref}[1]{Fig.\ref{#1}}
\newcommand{\tref}[1]{tab.\ref{#1}}
\newcommand{\Tref}[1]{Tab.\ref{#1}}
%
\newcommand{\nn}{\nonumber}
%
% Put the preprint numbers in the top right corner of the page.
% Use after \maketitle.
% First argument: How high it needs to be raised,
% Second argument: Width of the box,
% Third argument: The preprint numbers.
\newcommand{\preprintno}[3]{\hfill\raisebox{#1}[0cm][0cm]{
\begin{minipage}[t]{#2}\begin{flushright} #3 \end{flushright}\end{minipage}}
\vspace*{-\baselinestretch\baselineskip}}
%
% If you have changed the line spacing, e.g. with \renewcommand{\baselinestretch}{1.5},
% the command \sgap produces a line break with the normal spacing.
\newlength{\lingap}
\setlength{\lingap}{\baselinestretch\baselineskip}
\addtolength{\lingap}{-\baselineskip}
\newcommand{\sgap}{\\[-\lingap]}

%
\newcommand{\CD}{\mathcal{D}}
\newcommand{\CE}{\mathcal{E}}
\newcommand{\CG}{\mathcal{G}}
\newcommand{\CH}{\mathcal{H}}
\newcommand{\CK}{\mathcal{K}}
\newcommand{\CO}{\mathcal{O}}
\newcommand{\CL}{\mathcal{L}}
\newcommand{\CM}{\mathcal{M}}
\newcommand{\CN}{\mathcal{N}}
\newcommand{\CV}{\mathcal{V}}
\newcommand{\CZ}{\mathcal{Z}}
%
\newcommand{\dM}{\mathfrak{M}}
\newcommand{\dmd}{\mathfrak{d}}
\newcommand{\dmD}{\mathfrak{D}}
%
\newcommand{\R}{\mathbb{R}}
\newcommand{\C}{\mathbb{C}}
\newcommand{\CP}{\mathbb{CP}}
\newcommand{\Z}{\mathbb{Z}}
%
\newcommand{\ad}{{\dot{\alpha}}}
\newcommand{\bd}{{\dot{\beta}}}
\newcommand{\gd}{{\dot{\gamma}}}
\newcommand{\dd}{{\dot{\delta}}}
\newcommand{\ed}{{\dot{\epsilon}}}
%
\newcommand{\bs}{\overline{\sigma}}
\newcommand{\br}{\overline{\rho}}
\newcommand{\bpsi}{\overline{\psi}}
\newcommand{\bchi}{\overline{\chi}}
\newcommand{\bPsi}{\overline{\Psi}}
\newcommand{\bQ}{\overline{Q}}
\newcommand{\bS}{\overline{S}}
\newcommand{\bJ}{\overline{J}}
\newcommand{\zb}{{\bar z}}
\newcommand{\wb}{{\overline w}}
\newcommand{\cb}{{\bar c}}
\newcommand{\ab}{{\bar a}}
\newcommand{\bb}{{\bar b}}
\newcommand{\bp}{{\bar\partial}}
%
\newcommand{\p}{\partial}
\newcommand{\apm}{{\alpha^{\prime}}}
\newcommand{\adg}{a^\dagger}
\newcommand{\psq}{^{\prime\,2}}
\newcommand{\ppsq}{^{\prime\prime\,2}}
\newcommand{\half}{\frac{1}{2}}
%


\DeclareMathOperator{\cum}{cum}
%\DeclareMathOperator{\bias}{Bias}
%\DeclareMathOperator{\var}{Var}
\newcommand{\tk}{\tilde c}
\newcommand{\mh}{\hat\mu}
\newcommand{\sh}{\hat{s}}
\newcommand{\sih}{\hat{\sigma}}
%
%%%%%%%%%%%%%%%%%%%%%%%%%%%%%%%%%%%%%%%%%%%%%%%%%%%%%%%%%%%%%%%%%%%%%%%%%%
% Title info:
\title{Weighted estimators}
%
% Author List:
%
\author{Subhaneil Lahiri
\\
%
% Addresses:
%
\small{\emph{Harvard University}}
%
}

\begin{document}

\maketitle

% Preprint numbers, etc.
%\preprintno{8cm}{6cm}{
%    \texttt{arXiv:yymm.nnnn [hep-th]}
%}

%%%%%%%%%%%%%%%%%%%%%%%%%%%%%%%%%%%%%%%%%%%%%%%%%%%%%%%%%%%%%%%%%%%%%%%%%%


\begin{abstract}  We compute estimators and standard errors for samples that are independent but not identically distributed, assuming that cumulants scale in a particular way.
\end{abstract}


%%%%%%%%%%%%%%%%%%%%%%%%%%%%%%%%%%%%%%%%%%%%%%%%%%%%%%%%%%%%%%%%%%%%%%%%%%
% Beginning of Article:
%%%%%%%%%%%%%%%%%%%%%%%%%%%%%%%%%%%%%%%%%%%%%%%%%%%%%%%%%%%%%%%%%%%%%%%%%%

\section{Grouping data points}

Suppose that we start with an independent, identically distributed (iid) sample, \ie a set of random variables $\{y_a\}$, $a=1...M$
%
\begin{equation}\label{eq:iid}
  \cum_r(y_a) = c_r.
\end{equation}
%
Now, say we put them into groups of size $\{n_i\}$, $i=1...N$, and only recorded the means of each group
%
\begin{equation}\label{eq:group}
  x_i = \frac{\sum_{b=1}^{n_i}y_{n_{i-1}+b}}{n_i}\,.
\end{equation}
%
Then the $\{n_i\}$ would be independent but not identically distributed. In fact, their cumulants would be
%
\begin{equation}\label{eq:groupcum}
  \cum_r(x_i) = \frac{c_r}{(n_i)^{r-1}}\,.
\end{equation}
%
In practice, we often do not have any concept of ``number of samples per group''. Instead we have some quantity that scales the same way, \ie
%
\begin{equation}\label{eq:scalecum}
  \cum_r(x_i) = \frac{\tk_r}{(w_i)^{r-1}}\,.
\end{equation}
%
We will refer to this as independent, differently scaled (ids). One can think of these quantities as being related to \eqref{eq:groupcum} by
%
\begin{equation}\label{eq:scale}
  w_i = \alpha n_i, \qquad \tk_r = \alpha^{r-1} c_r.
\end{equation}
%
Note the $w_i$ and $\tk_r$ are not separately well defined, as we can redefine them (changing $\alpha$ in \eqref{eq:scale}) with,
%
\begin{equation}\label{eq:rescale}
  w_i \to \lambda w_i, \qquad \tk_r \to \lambda^{r-1} \tk_r.
\end{equation}
%
Only quantities that are invariant under this redefinition are well defined. In particular, $\tk_1$ is the only one of the $\tk_r$ that is really meaningful.

\section{Estimators}

First let us define
%
\begin{equation}\label{eq:defcum}
  \mu = \tk_1,\quad
  \sigma^2 = \tk_2, \quad
  \gamma = \tk_3, \quad
  \kappa = \tk_4.
\end{equation}
%
As the $x_i$ are independent, we have
%
\begin{equation}\label{eq:polyexp}
  \begin{aligned}
    \E(x_i) =& \mu,\\
    \E(x_i x_j) =& \mu^2
        + \sigma^2 \frac{\delta_{ij}}{w_i}\,,\\
    \E(x_i x_j x_k) =& \mu^3
        + \mu\sigma^2 \prn{\frac{\delta_{ij}}{w_i} +\frac{\delta_{jk}}{w_j} +\frac{\delta_{ki}}{w_k}}
        + \gamma \frac{\delta_{ijk}}{w_i^2}\,,\\
    \E(x_i x_j x_k x_l) =& \mu^4
        + \mu^2\sigma^2 \prn{\frac{\delta_{ij}}{w_i} +\frac{\delta_{ik}}{w_i} +\frac{\delta_{il}}{w_i} +\frac{\delta_{jk}}{w_j} +\frac{\delta_{jl}}{w_j} +\frac{\delta_{kl}}{w_k}}
        \\ &+ \sigma^4  \prn{\frac{\delta_{ij}}{w_i}\frac{\delta_{kl}}{w_k} +\frac{\delta_{ik}}{w_i}\frac{\delta_{jl}}{w_j} +\frac{\delta_{il}}{w_i}\frac{\delta_{jk}}{w_j}}
        \\ &+ \mu\gamma \prn{\frac{\delta_{ijk}}{w_i^2} +\frac{\delta_{ijl}}{w_j^2} +\frac{\delta_{ikl}}{w_k^2} +\frac{\delta_{jkl}}{w_l^2}}
        + \kappa \frac{\delta_{ijkl}}{w_i^3}\,,
  \end{aligned}
\end{equation}
%
where the generalised Kronecker-delta symbols are defined by
%
\begin{equation}\label{eq:kronecker}
  \delta_{ijkl...} = \left\{
  \begin{aligned}
    1\quad &\text{if}\quad i=j=k=l...\\
    0\quad &\text{otherwise}.
  \end{aligned}
  \right.
\end{equation}
%

We will now go through the construction of an estimator for the mean, $\mu$, in detail. Let
%
\begin{equation}\label{eq:estmeangen}
  \mh = \sum_i g_i x_i.
\end{equation}
%
We have
%
\begin{equation}\label{eq:estmeanbias}
  \bias(\mh) = \prn{\sum_i g_i -1} \mu, \qquad
  \var(\mh) = \prn{\sum_i \frac{g_i^2}{w_i}} \sigma^2.
\end{equation}
%
If we minimise the variance subject to the constraint that the bias is zero,
%
\begin{equation}\label{eq:minvarmean}
  \pdiff{\var(\mh)}{g_i} - \beta \pdiff{\bias(\mh)}{g_i} = \frac{2g_i}{w_i}\sigma^2 - \beta\mu = 0,
\end{equation}
%
where $\beta$ is a Lagrange multiplier. Therefore, we must set
%
\begin{equation}\label{eq:meanweights}
  g_i = \frac{w_i}{\sum_j w_j}\,.
\end{equation}
%
This leaves us with
%
\begin{equation}\label{eq:meanest}
  \mh = \frac{\sum_i w_i x_i}{\sum_i w_i}, \qquad
  \bias(\mh) = 0, \qquad
  \var(\mh) = \frac{\sigma^2}{\sum_i w_i}\,.
\end{equation}
%
Therefore, this estimator is unbiased and consistent, assuming that the $w_i/\sigma^2$ do not decrease faster than the sample size increases.

We will construct an estimator for the variance in much less detail, starting with a good guess and improving it. Consider the quantity
%
\begin{equation}\label{eq:varestguess}
  \sh^2 = \frac{\sum_i w_i (x_i-\mh)^2}{\sum_i w_i}\,.
\end{equation}
%
Using \eqref{eq:polyexp}, one can show that
%
\begin{equation}\label{eq:varestmoments}
 \begin{aligned}
  \E(\sh^2) &= \frac{N-1}{\sum_i w_i} \sigma^2, \\
  \var(\sh^2) &= \frac{2(N-1)}{\prn{\sum_i w_i}^2} \sigma^2 + \frac{\prn{\sum_i w_i^{-1}}\prn{\sum_i w_i} - 2N + 1}{\prn{\sum_i w_i}^3}\kappa.
 \end{aligned}
\end{equation}
%
So we can define an estimator for the variance
%
\begin{equation}\label{eq:varest}
 \begin{aligned}
  \sih^2 &= \frac{\sum_i w_i (x_i-\mh)^2}{N-1}, \qquad&
  \bias(\sih^2) &= 0, \\ & &
  \var(\sih^2) &= \frac{2\sigma^4}{N-1} + \frac{\brk{\prn{\sum_{ij} w_i/w_j} - 2N + 1}\kappa}{\prn{\sum_i w_i}(N-1)^2}\,.
 \end{aligned}
\end{equation}
%
This estimator is unbiased and consistent, as above. However it is not invariant under the rescaling \eqref{eq:rescale}. This is not surprising, given that $\sigma^2$ is itself not invariant under a rescaling of all the weights. Nevertheless, it can be used to compute a standard error in the mean:
%
\begin{equation}\label{eq:stderr}
  \delta\mu^2 = \frac{\sum_i w_i (x_i-\mh)^2}{\prn{\sum_i w_i}(N-1)}, \qquad
  \E(\delta\mu^2) = \var(\mh).
\end{equation}
%
Not that both the estimators for the mean \eqref{eq:meanest} and its standard error \eqref{eq:stderr} are invariant under the rescaling \eqref{eq:rescale}. These are the only two fully meaningful formulae in the section.

\section{One sample T-test}\label{sec:ttestone}

In this section we will assume that all cumulants except for the first two vanish, \ie that the distributions are normal
%
\begin{equation}\label{eq:scalednormal}
  x_i \sim \text{N}\prn{\mu,\frac{\sigma^2}{w_i}}. 
\end{equation}
%
We wish to test the hypothesis that the mean is equal to some value, $\mu_0$. Let
%
\begin{equation}\label{eq:stdvars}
  z_i \equiv \frac{\sqrt{w_i}}{\sigma}(x_i-\mu_0),
  \qquad
  Z = \sum_i z_i^2.
\end{equation}
%
Under the null hypothesis that $\mu=\mu_0$,
%
\begin{equation}\label{eq:stddists}
  z_i \sim \text{N}(0,1),
  \qquad
  Z \sim \chi^2_N.
\end{equation}
%
We can rewrite $Z$ as
%
\begin{equation}\label{eq:cochranversion}
  \begin{aligned}
    Z &= \frac{1}{\sigma^2} \sum_i w_i (x_i - \mu_0)^2 \\
      &= \frac{1}{\sigma^2} \sum_i w_i (x_i - \mh + \mh - \mu_0)^2 \\
      &= \frac{1}{\sigma^2} \sum_i w_i \brk{(x_i - \mh)^2 + (\mh - \mu_0)^2} \\
      &= \prn{\frac{\sum_i w_i}{\sigma^2}}  \brk{(N-1)\delta\mh^2 + (\mh - \mu_0)^2}.
  \end{aligned}
\end{equation}
%
This can be rewritten back in terms of the $z_i$ as
%
\begin{equation}\label{eq:cochrancoeffs}
  \begin{aligned}
    \prn{\frac{\sum_i w_i}{\sigma^2}}  (N-1)\delta\mh^2 &= \sum_{ij} S_{ij} z_i z_j, 
    &\qquad
     S_{ij} &= \delta_{ij} - \frac{\sqrt{w_i w_j}}{\sum_k w_k}, \\
    \prn{\frac{\sum_i w_i}{\sigma^2}}  (\mh - \mu_0)^2  &= \sum_{ij} M_{ij} z_i z_j,
    &\qquad
     M_{ij} &= \frac{\sqrt{w_i w_j}}{\sum_k w_k}.
  \end{aligned}
\end{equation}
%
We can see that $M_{ij}$ is the projection operator onto a vector of the square root of the weights and $S_{ij}$ is the perpendicular projection operator, therefore
%
\begin{equation}\label{eq:cochranrank}
  \rank(S_{ij}) = N-1, \qquad \rank(M_{ij}) = 1.
\end{equation}
%
By Cochran's theorem \cite{Cochran:1934},
%
\begin{equation}\label{eq:cochrandist}
  \begin{aligned}
    \prn{\frac{\sum_i w_i}{\sigma^2}}  (N-1)\delta\mh^2 &\sim \chi^2_{N-1}, \\
    \prn{\frac{\sum_i w_i}{\sigma^2}}  (\mh - \mu_0)^2  &\sim \chi^2_{1},
  \end{aligned}
\end{equation}
%
and these two quantities are independent. This allows us to conclude that
%
\begin{equation}\label{eq:Tstats}
  \begin{aligned}
    t^2 &\equiv \frac{(\mh - \mu_0)^2}{\delta\mh^2} &&\sim \text{F}_{1,N-1}, \\
    t   &\equiv \frac{\mh - \mu_0}{\delta\mh}       &&\sim \text{T}_{N-1}.
  \end{aligned}
\end{equation}
%
Therefore, we can do one sample T-tests in the same way as usual.




%\section*{Acknowledgements}



%%%%%%%%%%%%%%%%%%%%%%%%%%%%%%%%%%%%%%%%%%%%%%%%%%%%%%%%%%%%%%%%%%%%%%%%%%
%\section*{Appendices}
%\appendix
%%%%%%%%%%%%%%%%%%%%%%%%%%%%%%%%%%%%%%%%%%%%%%%%%%%%%%%%%%%%%%%%%%%%%%%%%%





%%%%%%%%%%%%%%%%%%%%%%%%%%%%%%%%%%%%%%%%%%%%%%%%%%%%%%%%%%%%%%%%%%%%%%%%%%

\bibliographystyle{utcaps_sl}
\bibliography{maths}

\end{document}
